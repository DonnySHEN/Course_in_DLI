%  author: Donny SHEN 
%  Class: MATH 2102
%  Complete Date: 2025.01.21
%  Last Modification: 2025.01.21

\documentclass[12pt]{ctexart}

% 宏包加载
\usepackage{tikz}
\usepackage{newtxtext, geometry, amsmath, amssymb}
\usepackage{amsthm}
\usepackage{xpatch}
\usepackage[strict]{changepage}
\usepackage{multicol}
\usepackage{pdfpages}
\usepackage[fontsize=14pt]{fontsize}
\usepackage{esvect}
\usepackage{xcolor} % 加载颜色宏包
\usepackage{tocloft} % 自定义目录样式
\usepackage{tcolorbox}
\usepackage{graphicx}
\usepackage{hyperref}
\hypersetup{
    colorlinks=false,  % 禁用超链接的颜色
    pdfborder={0 0 0}  % 移除超链接的边框
}
\newtheorem{proposition}{命题}

\tcbuselibrary{most}
% 加载文献引用所需宏包
\usepackage[backend=biber,style=gb7714-2015,citestyle=numeric-comp,gbalign=gb7714-2015]{biblatex}
\addbibresource{references.bib}
\newtheorem*{lemma}{引理}
\newtheorem*{theorem}{定理}
\geometry{
    a4paper,         % 使用 A4 纸
    left=1.5cm,        % 左边距
    right=1.5cm,       % 右边距
    top=2.5cm,       % 上边距
    bottom=2.5cm     % 下边距
}

\renewcommand{\contentsname}{\hfill\centering\zihao{2} 目录\hfill~}

\begin{document}
\title{数学分析拓展3大作业} % 文档标题
\newpage
\tableofcontents % 生成目录
\newpage


% 第一部分:作业题整理
\section{第一部分:作业题整理} % 无编号的 section
\setcounter{section}{1} % 手动设置章节编号为 1
\setcounter{subsection}{0} % 手动设置章节编号为 1
\subsection{集合与映射作业题} 
\subsubsection*{1.1.1}
试证:映射左可逆当且仅当它是单射;右可逆当且仅当它是满射

\begin{proof}
(1) 设 $f$ 映射左可逆,当且仅当它是单射。

\noindent
(a) 必要性:设 $g$ 是 $f$ 的左逆,则对 $\forall t \in T$,有 $g(f(t)) = t$。  
即 $g \circ f = I_T = I_T(t)$。有 $g \circ f = I_T$,而 $I_T$ 是双射,故 $f$ 是单射。

\noindent
(b) 充分性:定义 $g: S \to T$ 如下 $g(s) = t_1, \; s \in f(T)$ 且 $f(t_1) = s$。  
对每个 $s \in S$,$g(s)$ 只有一个值,且若 $f(t_1) = s$,因 $g \circ f(t_1) = g(s) = t$。

\noindent
故 $g$ 是 $f$ 的左逆。

(2) 右可逆当且仅当它是满射。

\noindent
(a) 必要性:设 $g$ 是 $f$ 的右逆,则对 $\forall s \in S$,有 $f(g(s)) = s$。  
有 $f \circ g = I_S$,而 $I_S$ 是双射,故 $f$ 是满射。

\noindent
(b) 充分性:若 $f$ 是满射,则对任意 $s \in S$,至少存在一个 $t \in T$,使得 $f(t) = s$。  
定义 $g: S \to T$ 如下,对每个 $s \in S$,有:
\begin{enumerate}
    \item 若只有一个 $t \in T$ 使得 $f(t) = s$,令 $g(s) = t$。
    \item 若有 $t_1, t_2, \ldots, t_n \in T$,使得 $f(t_1) = f(t_2) = \cdots = f(t_n) = s$,则取某一个 $t_i$,令 $g(s) = t_i$。
\end{enumerate}
这样,对每个 $s \in S$,$g(s)$ 只有一个值,且 $f(g(s)) = s$。  
故 $g$ 是 $f$ 的右逆。

\end{proof}

\subsubsection*{1.1.2} 
试证:全体有理数是可数的。
\begin{proof}
我们先证明 $[0, 1)$ 中全体有理数是可数的。显然,排列
\[
0, \; \frac{1}{2}, \; \frac{1}{3}, \; \frac{2}{3}, \; \frac{1}{4}, \; \frac{2}{4}, \;\frac{3}{4}, \ldots
\]
这样就列举出了 $[0, 1)$ 中的所有有理数。我们把这些数重新排列:从第一行开始,从左到右排列成
\[
0, \; \frac{1}{2}, \; \frac{1}{3}, \; \frac{2}{3}, \; \frac{1}{4}, \; \frac{3}{4}, \ldots
\]
然后剔除重复的数,可得列出的就是 $[0, 1)$ 中有理数的全集。

很明显,当有理数 $r \in [0, 1)$ 时,对任何整数 $n$,映射 $r \to r + n$ 是 $[0, 1)$ 中的有理数和 $[n, n+1)$ 中的有理数之间的一一对应。  
因此,$[n, n+1)$ 的全体有理数也是可数的。

这样,上述全体有理数可以表示为
\[
\bigcup_{n=-\infty}^{\infty} \{x : x \in [n, n+1), \; x \in \mathbb{Q}\},
\]
这是可数个互不相交的可数集的并。

因此,全体有理数是可数的。
\end{proof}

\subsubsection*{1.1.3} 
试证:可数个至多可数集的并集是至多可数的。
\begin{proof}
设意即为 $S = \bigcup E_n \; (n = 1, 2, 3, \ldots)$ 是一列至多可数集。

令 $S = \bigcup E_n$,那么 $S$ 是至多可数集。  
不妨对每个 $n \in \mathbb{N}^*$,$E_n = \{x_{n1}, x_{n2}, \ldots, x_{nk}, \ldots\}$。

考虑下列无限矩阵列:
\[
\begin{array}{ccccccc}
x_{11}, & x_{12}, & x_{13}, & x_{14}, & \ldots \\
x_{21}, & x_{22}, & x_{23}, & x_{24}, & \ldots \\
x_{31}, & x_{32}, & x_{33}, & x_{34}, & \ldots \\
x_{41}, & x_{42}, & x_{43}, & x_{44}, & \ldots \\
\vdots & \vdots & \vdots & \vdots & \ddots \\
\end{array}
\]

其中第 $n$ 行由 $E_n$ 的元素组成,这个矩阵列包含 $S$ 中的所有元素。

按照简单指示图的那样,这些元素可以排成一行:
\[
x_{11}, \; x_{12}, \; x_{21}, \; x_{13}, \; x_{22}, \; x_{31}, \; x_{14}, \; x_{23}, \; x_{32}, \; x_{41}, \; \ldots
\]

当两个集合 $E_i$ 和 $E_j$ 有公共元素时,这些元素在这一行中会重复出现。  
我们从左到右顺次,重复元素仅保留第一次出现的那个。

这样以后,得到的单集合 $S$ 即为 $S$ 的所有元素。  
因此,$S$ 是至多可数的。
\end{proof}

\subsubsection*{1.1.4} 
试构造一个开区间 $(0, 1)$ 与闭区间 $[0, 1]$ 之间的一一对应。
\begin{proof}
我们构造下面函数从 $[0, 1]$ 到 $(0, 1)$ 上的映射 $f$:

\[
f(x) =
\begin{cases} 
\frac{1}{2}, & \text{当 } x = 0 \text{ 时}, \\
\frac{1}{n+2}, & \text{当 } x = \frac{1}{n} \text{ 时 } (n \in \mathbb{N}^*), \\
x, & \text{当 } x \notin \{\frac{1}{n} \mid n \in \mathbb{N}^*\} \text{ 时}.
\end{cases}
\]

显然,$f$ 是 $(0, 1)$ 与 $[0, 1]$ 之间的一一对应关系。
\end{proof}

\subsubsection*{1.1.5} 
试证:每个无限集与自身的一个真子集对等。
\begin{proof}
因为有限集是不存在自反真子集对等的,若证明 $A$ 为无限集的充要条件为 $A$ 与其某真子集对等,即可证明题目。

\begin{enumerate}
    \item 充分性:因为有限集不存在真子集对等,故充要性显然成立。
    \item 必要性:取 $A$ 中一个非空有限子集 $B$。显然,有 $A \sim (A \setminus B)$,即得必要性。
\end{enumerate}

故每个无限集都与自身的一个真子集对等。
\end{proof}

\subsection{数列极限作业题}
\subsubsection*{1.2.1} 
设 $a > 0$。求极限 $\lim_{n \to \infty} \sqrt[n]{a}$ 和 $\lim_{n \to \infty} \frac{a^n}{n!}$。
\begin{proof}
\textbf{(1) 证明 $\lim_{n \to \infty} \sqrt[n]{a} = 1$:}

1. 当 $a = 1$ 时,显然存在 $\lim_{n \to \infty} \sqrt[n]{a} = \lim_{n \to \infty} \sqrt[n]{1} = 1$。

2. 当 $a > 1$ 时,记 $a^{1/n} = 1 + d_n$,则 $d_n > 0$。  
由 $a = (1 + d_n)^n = 1 + n d_n + \frac{n(n-1)}{2} d_n^2 + \cdots$,得:
\[
d_n = a^{1/n} - 1 \leq \frac{\ln a}{n}.
\]
显然,当 $n \to \infty$ 时,$a^{1/n} \to 1$。

3. 当 $0 < a < 1$ 时,令 $\frac{1}{a} > 1$,同理可得 $\lim_{n \to \infty} \sqrt[n]{a} = 1$。

综上,$\lim_{n \to \infty} \sqrt[n]{a} = 1$。

\textbf{(2) 证明 $\lim_{n \to \infty} \frac{a^n}{n!} = 0$:}

1. 当 $a \neq 0$ 时,设 $k = \lfloor |a| \rfloor + 1$,表示不大于 $|a|$ 的整数加 $1$,  
则:
\[
\frac{a^n}{n!} = \frac{|a|^n}{1 \cdot 2 \cdot \cdots \cdot n} \leq \frac{k}{n \cdot n \cdot \cdots \cdot n}.
\]

2. 对于任意给定的 $\varepsilon > 0$,取 $N = \max\{k, \frac{k |a|}{\varepsilon}\}$,  
当 $n > N$ 时,有:
\[
\frac{a^n}{n!} \leq k \cdot \frac{|a|}{n!} < \varepsilon.
\]

故 $\lim_{n \to \infty} \frac{a^n}{n!} = 0$。

\textbf{综上:}
\[
\lim_{n \to \infty} \sqrt[n]{a} = 1, \quad \lim_{n \to \infty} \frac{a^n}{n!} = 0.
\]
\end{proof}

\subsubsection*{1.2.2} 
已知 $\lim_{n \to \infty} a_n = a$,试证
\[
\lim_{n \to \infty} \frac{a_1 + a_2 + \cdots + a_n}{n} = \lim_{n \to \infty} \frac{\lfloor na_n \rfloor}{n} = a,
\]
此处方括号表示取整函数。
\begin{proof}
\textbf{(1) 证明 $\lim_{n \to \infty} \frac{a_1 + a_2 + \cdots + a_n}{n} = a$:}

已知 $\lim_{n \to \infty} a_n = a$,即对于任意 $\varepsilon > 0$,存在 $m \in \mathbb{N}$,当 $n > m$ 时,有 $|a_n - a| < \varepsilon$。

取自然数 $m$,设 $|a_1 - a| + |a_2 - a| + \cdots + |a_m - a| = A$ 是正数,  
已知 $\lim_{n \to \infty} \frac{A}{n} = 0$,即对于上述同样的 $\varepsilon > 0$,存在 $N \in \mathbb{N}^*$ 且 $N > m$,当 $n > N$ 时,有 $\frac{A}{n} < \varepsilon$。

从而有:
\[
\left| \frac{a_1 + a_2 + \cdots + a_n}{n} - a \right| = \left| \frac{(a_1 - a) + (a_2 - a) + \cdots + (a_n - a)}{n} \right| 
\] 
\[
\leq \frac{|a_1 - a| + |a_2 - a| + \cdots + |a_m - a|}{n} + \frac{|a_{m+1} - a| + \cdots + |a_n - a|}{n}.
\]
前一项小于 $\frac{A}{n}$,后一项小于 $\varepsilon$,因此:
\[
\frac{A}{n} + \frac{n-m}{n} \varepsilon < \varepsilon + \varepsilon = 2\varepsilon。
\]
由 $\varepsilon$ 的任意性,可得 $\lim_{n \to \infty} \frac{a_1 + a_2 + \cdots + a_n}{n} = a$。

\textbf{(2) 证明 $\lim_{n \to \infty} \frac{\lfloor na_n \rfloor}{n} = a$:}

因为 $|\lfloor na_n \rfloor - na_n| < 1$,故:
\[
\lim_{n \to \infty} \frac{\lfloor na_n \rfloor - na_n}{n} = 0。
\]

又有:
\[
\left| \frac{\lfloor na_n \rfloor}{n} - a \right| = \left| \frac{\lfloor na_n \rfloor - na_n}{n} + \frac{na_n}{n} - a \right|
\leq \frac{|\lfloor na_n \rfloor - na_n|}{n} + |a_n - a|。
\]

因为 $\lim_{n \to \infty} a_n = a$,有 $\lim_{n \to \infty} |a_n - a| = 0$,从而:
\[
\lim_{n \to \infty} \frac{\lfloor na_n \rfloor}{n} = a。
\]

\textbf{综上:}
\[
\lim_{n \to \infty} \frac{a_1 + a_2 + \cdots + a_n}{n} = \lim_{n \to \infty} \frac{\lfloor na_n \rfloor}{n} = a。
\]
\end{proof}

\subsubsection*{1.2.3} 
设 $a_0 + a_1 + \cdots + a_p = 0$。试证:
\[
\lim_{n \to \infty} \left( a_0 \sqrt{n} + a_1 \sqrt{n+1} + \cdots + a_p \sqrt{n+p} \right) = 0。
\]
\begin{proof}
注意到:
\[
\sum_{k=0}^p a_k \sqrt{n+k} = \sum_{k=0}^p \left( a_0 + a_1 + \cdots + a_k \right) (\sqrt{n+k} - \sqrt{n+k+1}),
\]
\[
= \sum_{k=0}^p \frac{a_0 + a_1 + \cdots + a_k}{\sqrt{n+k} + \sqrt{n+k+1}}.
\]

由于 $a_0 + a_1 + \cdots + a_p = 0$,显然:
\[
\lim_{n \to \infty} \left( a_0 \sqrt{n} + a_1 \sqrt{n+1} + \cdots + a_p \sqrt{n+p} \right) = \lim_{n \to \infty} 0 = 0。
\]
\end{proof}

\subsubsection*{1.2.4} 
设数列 $\{a_n\}_{n=1}^\infty$ 满足 $\lim_{n \to \infty} (a_n - a_{n-2}) = 0$,试证:
\[
\lim_{n \to \infty} \frac{a_n - a_{n-1}}{n} = 0。
\]
\begin{proof}
为方便起见,令 $a_0 = 0$,那么:
\[
\lim_{n \to \infty} \frac{a_{2n}}{n} = \lim_{n \to \infty} \frac{(a_2 - a_0) + (a_4 - a_2) + \cdots + (a_{2n} - a_{2n-2})}{n}.
\]

由已知条件 $\lim_{n \to \infty} (a_{2n} - a_{2n-2}) = 0$,因此:
\[
\lim_{n \to \infty} \frac{a_{2n}}{n} = 0。
\]

同理,可得:
\[
\lim_{n \to \infty} \frac{a_{2n+1}}{2n+1} = 0。
\]

合并后得:
\[
\lim_{n \to \infty} \frac{a_n}{n} = 0。
\]

进一步,计算:
\[
\lim_{n \to \infty} \frac{a_n - a_{n-1}}{n} = \lim_{n \to \infty} \left( \frac{a_n}{n} - \frac{a_{n-1}}{n-1} \cdot \frac{n-1}{n} \right)。
\]

由 $\lim_{n \to \infty} \frac{a_n}{n} = 0$ 和 $\lim_{n \to \infty} \frac{n-1}{n} = 1$,有:
\[
\lim_{n \to \infty} \frac{a_n - a_{n-1}}{n} = 0 - 0 = 0。
\]

综上,$\lim_{n \to \infty} \frac{a_n - a_{n-1}}{n} = 0$。
\end{proof}

\subsubsection*{1.2.5} 
设数列 $\{a_n\}_{n=1}^\infty$ 满足 $0 < a_n < 1$,$(1 - a_n)a_{n+1} > \frac{1}{4}$,$n \in \mathbb{N}$。

试证:
\[
\lim_{n \to \infty} a_n = \frac{1}{2}。
\]

\begin{proof}
注意到函数 $(1 - x)x$ 在 $[0, 1]$ 上的最大值为 $\frac{1}{4}$,  
因此由条件 $(1 - a_n)a_{n+1} > \frac{1}{4} \geq (1 - a_n)a_n$,可知 $\{a_n\}$ 是严格单调递增的。

进而可得 $\{a_n\}$ 有极限,令其为 $a$,那么有:
\[
(1 - a)a \geq \frac{1}{4}。
\]

解得 $a = \frac{1}{2}$。

因此,$\lim_{n \to \infty} a_n = a = \frac{1}{2}$。
\end{proof}

\subsubsection*{1.2.6} 
设数列 $\{a_n\}_{n=1}^\infty$ 对所有正整数 $n, p$ 满足
\[
|a_{n+p} - a_n| \leq \frac{p}{n},
\]
问 $\{a_n\}_{n=1}^\infty$ 是否是基本列?
\begin{proof}
$\{a_n\}$ 不一定是基本列。例如取 $a_n = 1 + \frac{1}{2} + \frac{1}{3} + \cdots + \frac{1}{n}$,  
那么 $|a_{n+p} - a_n| \leq \frac{p}{n}$ 是成立的,但 $\{a_n\}$ 显然发散,不是基本列。
\end{proof}

\subsubsection*{1.2.7} 
设数列 $\{a_n\}_{n=1}^\infty$ 对所有正整数 $n, p$ 满足
\[
|a_{n+p} - a_n| \leq \frac{p}{n^2},
\]
问 $\{a_n\}_{n=1}^\infty$ 是否是基本列?

\begin{proof}
$\{a_n\}$ 是基本列。

对 $\varepsilon > 0$,只要取 $N = \lfloor \frac{1}{\varepsilon} \rfloor + 1$,那么当 $n > N$ 时,有:
\[
|a_{n+p} - a_n| \leq |a_{n+p} - a_{n+p-1}| + |a_{n+p-1} - a_{n+p-2}| + \cdots + |a_{n+1} - a_n|
\]
\[
\leq \frac{1}{(n+p-1)^2} + \frac{1}{(n+p-2)^2} + \cdots + \frac{1}{n^2}
\]
\[
\leq \frac{1}{n(n+1)} + \frac{1}{(n+1)(n+2)} + \cdots + \frac{1}{(n+p-1)n}
\]
\[
= \frac{1}{n(n-1)} < \varepsilon。
\]

即 $\{a_n\}$ 是柯西收敛级列,故 $\{a_n\}$ 是基本列。
\end{proof}

\subsubsection*{1.2.8} 
设数列 $\{a_n\}_{n=1}^\infty$ 有界且发散,试证 $\{a_n\}_{n=1}^\infty$ 必有两个子列收敛于不同的极限。
\begin{proof}
\begin{enumerate}
    \item 由列紧性定理知 $\{a_n\}$ 有一个收敛子列,极限设为 $a$。
    \item 因为 $\{a_n\}$ 不以 $a$ 为极限,所以 $\exists \varepsilon_0 > 0$,使得对于每行 $j > 0$,都存在 $k_j > 0$ 使得 $|a_{k_j} - a| \geq \varepsilon_0$。
    \item 不妨让 $k_{j+1} > k_j$,那么从子列 $\{a_{k_j}\}$ 中又可以取出另一个收敛子列。显然,这是 $\{a_n\}$ 的一个收敛子列,且不以 $a$ 为极限。
\end{enumerate}

综上,$\{a_n\}$ 有界发散,$\{a_n\}$ 必有两个子列收敛于不同的极限。
\end{proof}

\subsection{有限覆盖上下极限作业题}
\subsubsection*{1.3.1} 
设 $\{a_n\}_{n=1}^\infty, \{b_n\}_{n=1}^\infty$ 为有界数列,试证:
\[
\liminf a_n + \liminf b_n \leq \liminf (a_n + b_n) 
\leq \liminf a_n + \limsup b_n \leq \limsup a_n + \limsup b_n。
\]
\begin{proof}
注意到上下极限的关系,要证的是不等式,仅需证明:
\[
\liminf a_n + \liminf b_n \leq \liminf (a_n + b_n) 
\leq \liminf a_n + \limsup b_n \tag{1}
\]
\[
\liminf (a_n + b_n) 
\leq \liminf a_n + \limsup b_n \leq \limsup a_n + \limsup b_n \tag{2}
\]


我们只需证明其中的不等式 $(1)$ 和 $(2)$,其余不等式可以由类似推导得到。

\textbf{证明不等式 $(1)$:}
\[
\inf a_k + \inf b_k \leq \inf (a_k + b_k) \leq \inf a_k + \sup b_k。
\]

1. 当 $k \to \infty$ 时:
\[
\inf a_k \leq a_k, \quad \inf b_k \leq b_k \implies a_k + b_k \geq \inf a_k + \inf b_k。
\]
因此,$\inf a_k + \inf b_k$ 是 $a_k + b_k$ 的一个下界,从而:
\[
\inf a_k + \inf b_k \leq \inf (a_k + b_k)。 \tag{3}
\]

2. 记 $c_k = a_k + b_k$,则 $a_k = c_k - b_k$。由此可得:
\[
\inf a_k = \inf (c_k - b_k) \geq \inf c_k - \sup b_k。
\]
根据 $\inf c_k = \inf (a_k + b_k)$,因此:
\[
\inf (a_k + b_k) \leq \inf a_k + \sup b_k。 \tag{4}
\]

由 $(3)$ 和 $(4)$,可得:
\[
\inf a_k + \inf b_k \leq \inf (a_k + b_k) \leq \inf a_k + \sup b_k。 \tag{5}
\]

\textbf{证明不等式 $(2)$:} 类似证明,可以得到:
\[
\liminf a_n + \liminf b_n \leq \liminf (a_n + b_n) \leq \limsup a_n + \limsup b_n。
\]

综上,证明了所需的不等式:
\[
\liminf a_n + \liminf b_n \leq \liminf (a_n + b_n) \leq \liminf a_n + \limsup b_n \leq \limsup a_n + \limsup b_n。
\]
\end{proof}

\subsubsection*{1.3.2} 
设 $\{a_n\}_{n=1}^\infty$ 是正数列,且 $\liminf a_n > 0$。试证:
\[
\limsup \frac{1}{a_n} = \frac{1}{\liminf a_n}, \quad \liminf \frac{1}{a_n} = \frac{1}{\limsup a_n}。
\]
\begin{proof}
1. 证明 $\limsup \frac{1}{a_n} = \frac{1}{\liminf a_n}$:

(1) 存在 $\{a_{n_k}\}$ 的子列,使得:
\[
\limsup \frac{1}{a_n} = \lim \frac{1}{a_{n_k}} = \frac{1}{\liminf a_n}。
\]

(2) 又存在另一子列 $\{a_{m_k}\}$,使得:
\[
\liminf \frac{1}{a_n} = \lim \frac{1}{a_{m_k}} = \frac{1}{\limsup a_n}。
\]

因此:
\[
\limsup \frac{1}{a_n} = \frac{1}{\liminf a_n}。
\]

2. 证明 $\liminf \frac{1}{a_n} = \frac{1}{\limsup a_n}$:

由上述结论 $\limsup \frac{1}{a_n} = \frac{1}{\liminf a_n}$,可得:
\[
(\limsup \frac{1}{a_n}) \cdot (\liminf a_n) = 1。
\]

因为 $\{a_n\}$ 是正数列,且 $\liminf a_n > 0$,显然有:
\[
(\limsup a_n) \cdot (\liminf \frac{1}{a_n}) = 1。
\]

由 $\{a_n\}$ 的任意性,故有:
\[
\liminf \frac{1}{a_n} = \frac{1}{\limsup a_n}。
\]
\end{proof}

\subsubsection*{1.3.3} 
设 $\{a_n\}_{n=1}^\infty$ 为正数列,且
\[
(\limsup a_n) \left(\limsup \frac{1}{a_n}\right) = 1。
\]
试证 $\{a_n\}_{n=1}^\infty$ 收敛。

\begin{proof}
易知 $\{a_n\}$ 为正数列,故 $\liminf a_n \geq 0$。

由题意可知:
\[
(\limsup a_n) \left(\limsup \frac{1}{a_n}\right) = 1,
\]
得:
\[
\limsup \frac{1}{a_n} = \frac{1}{\limsup a_n}。
\]

若 $\liminf a_n > 0$,则显然有 $\liminf a_n = \limsup a_n$,故 $\{a_n\}$ 收敛。

若 $\liminf a_n = 0$,那么:
\[
\limsup \frac{1}{a_n} = +\infty。
\]
同理 $\limsup a_n = 0$。因此 $\{a_n\}$ 收敛。

综上,$\{a_n\}$ 收敛。
\end{proof}

\subsubsection*{1.3.4} 
设 $\{a_n\}_{n=1}^\infty$ 为正数列,试证:
\[
\limsup \sqrt[n]{a_n} \leq \limsup \frac{a_{n+1}}{a_n}。
\]
\begin{proof}
记 $\alpha = \limsup \frac{a_{n+1}}{a_n}$。若 $\alpha = +\infty$,则不等式显然成立。

当 $0 \leq \alpha < +\infty$ 时,需证明 $\limsup \sqrt[n]{a_n} \leq \alpha$。  
只需证明对任意 $\varepsilon > 0$,有 $\limsup \sqrt[n]{a_n} < \alpha + \varepsilon$。

由定义,对于任意 $\varepsilon > 0$,存在 $N > 0$,当 $n > N$ 时,有:
\[
\frac{a_{n+1}}{a_n} < \alpha + \varepsilon。
\]

取 $n = N, N+1, \ldots$,则 $n-N$ 个相乘得:
\[
\frac{a_n}{a_N} = \frac{a_{N+1}}{a_N} \cdot \frac{a_{N+2}}{a_{N+1}} \cdots \frac{a_n}{a_{n-1}} < (\alpha + \varepsilon)^{n-N}。
\]

即有:
\[
a_n < a_N (\alpha + \varepsilon)^{n-N}。
\]

令 $M \equiv a_N (\alpha + \varepsilon)^{-N}$,则有:
\[
a_n < M (\alpha + \varepsilon)^n。
\]

因此:
\[
\sqrt[n]{a_n} < \sqrt[n]{M} (\alpha + \varepsilon),
\]
当 $n \to \infty$ 时,$\sqrt[n]{M} \to 1$,取上极限得:
\[
\limsup \sqrt[n]{a_n} \leq \alpha + \varepsilon。
\]

由于 $\varepsilon$ 的任意性,得:
\[
\limsup \sqrt[n]{a_n} \leq \alpha。
\]

综上,得:
\[
\limsup \sqrt[n]{a_n} \leq \limsup \frac{a_{n+1}}{a_n}。
\]
\end{proof}

\subsubsection*{1.3.5} 
给定数列 $\{a_n\}_{n=1}^\infty$,记
\[
b_n = 2a_n - 3a_{n+1}, \quad n = 1, 2, \ldots
\]
试证:数列 $\{b_n\}_{n=1}^\infty$ 收敛的充要条件是数列 $\{a_n\}_{n=1}^\infty$ 收敛。
\begin{proof}
\textbf{必要性}  
若 $\{a_n\}$ 收敛,不妨设 $\lim_{n \to \infty} a_n = a$,则:
\[
\lim_{n \to \infty} b_n = \lim_{n \to \infty} (2a_n - 3a_{n+1}) = 2a - 3a = a,
\]
显然 $\{b_n\}$ 收敛。

\textbf{充分性}  
若 $\{b_n\}$ 收敛,设 $\lim_{n \to \infty} b_n = b$。由定义:
\[
b_n = 2a_n - 3a_{n+1}。
\]
可得:
\[
a_{n+1} = \frac{2}{3}a_n - \frac{1}{3}b_n。
\]
令 $C_n = a_{n+1} - a_n$,作差得:
\[
C_{n+1} = \frac{2}{3}C_n - \frac{1}{3}(b_{n+1} - b_n)。
\]
当 $n \to \infty$ 时,因 $b_{n+1} - b_n \to 0$,可得:
\[
\lim_{n \to \infty} C_n = \frac{2}{3} \lim_{n \to \infty} C_n,
\]
显然 $\lim_{n \to \infty} C_n = 0$,即:
\[
\lim_{n \to \infty} (a_{n+1} - a_n) = 0。
\]

由 $\{a_n\}$ 的递推公式,可知 $\{a_n\}$ 收敛。

综上,$\{b_n\}$ 收敛的充要条件是 $\{a_n\}$ 收敛。
\end{proof}

\subsubsection*{1.3.6} 
设 $\alpha, \beta > 0, a_1 > 0$,且
\[
a_{n+1} = \alpha + \frac{\beta}{a_n}, \quad n = 1, 2, \ldots
\]
试证数列 $\{a_n\}_{n=1}^\infty$ 收敛,并求其极限。
\begin{proof}
由于 $a_1 > 0, \alpha > 0, \beta > 0$,由递推关系 $a_{n+1} = \alpha + \frac{\beta}{a_n}$ 可知,对于任意 $n \geq 1$,均有 $a_n > \alpha > 0$。

将递推公式写为:
\[
a_{n+1} = \alpha + \frac{\beta}{a_n}。
\]

定义函数 $f(x) = \alpha + \frac{\beta}{x}$,解方程 $f(x) = x$,得:
\[
x_1 = \frac{\alpha + \sqrt{\alpha^2 + 4\beta}}{2}, \quad x_2 = \frac{\alpha - \sqrt{\alpha^2 + 4\beta}}{2}。
\]

显然 $x_1 > 0$ 且 $x_2 < 0$,由于 $a_n > 0$,因此只能有 $a_n$ 收敛于 $x_1$。

令 $a_n - x_1 = b_n$,则:
\[
a_{n+1} - x_1 = f(a_n) - x_1。
\]

将 $f(x)$ 线性化,得:
\[
f(a_n) - x_1 = \frac{\beta}{a_n} - \frac{\beta}{x_1} = -\frac{\beta}{x_1^2}(a_n - x_1)。
\]

于是:
\[
b_{n+1} = -\frac{\beta}{x_1^2} b_n。
\]

显然,$|\frac{\beta}{x_1^2}| < 1$,因此 $b_n \to 0$,即 $a_n \to x_1$。

综上,$\{a_n\}$ 收敛,且极限为:
\[
\lim_{n \to \infty} a_n = \frac{\alpha + \sqrt{\alpha^2 + 4\beta}}{2}。
\]
\end{proof}

\subsubsection*{1.3.7} 
设数列 $\{a_n\}_{n=1}^\infty$ 满足 $a_{p+q} \leq a_p + a_q$,$p, q = 1, 2, \ldots$,试证:
\[
\lim_{n \to \infty} \frac{a_n}{n} = \inf_{n \geq 1} \frac{a_n}{n}。
\]

\begin{proof}
对于不等式 $a_{p+q} \leq a_p + a_q$,考虑其意义,便于想象,不妨先设 $p = 10$。  
此时,对于任意 $a_q$,例如 $a_{11} \leq a_{10} + a_1, a_{20} \leq 2a_{10} + a_2$。

一般地,对于任意自然数 $n$ 可以表示为 $n = k \cdot 10 + r$,其中 $k = 0, 1, 2, \ldots$,$r = 0, 1, 2, \ldots, 9$。  
从而有:
\[
a_n \leq k a_{10} + a_r。
\]
令 $k = \frac{n - r}{10} \to +\infty$,由此式可知:
\[
\inf_{n \geq 1} \frac{a_n}{n} \leq \frac{a_n}{n} \leq \frac{k a_{10}}{n} + \frac{a_r}{n},
\]
此式对一切 $n$ 成立。

令 $n \to +\infty$,取上下极限。注意 $k/n \to 1/10$,我们有:
\[
\inf_{n \geq 1} \frac{a_n}{n} \leq \liminf_{n \to \infty} \frac{a_n}{n} \leq \limsup_{n \to \infty} \frac{a_n}{n} \leq \inf_{n \geq 1} \frac{a_n}{n}。
\]
因此:
\[
\liminf_{n \to \infty} \frac{a_n}{n} = \limsup_{n \to \infty} \frac{a_n}{n} = \inf_{n \geq 1} \frac{a_n}{n}。
\]

综上:
\[
\lim_{n \to \infty} \frac{a_n}{n} = \inf_{n \geq 1} \frac{a_n}{n}。
\]
\end{proof}

\subsubsection*{1.3.8}
设 $\{a_n\}_{n=1}^\infty$ 为正数列,试证:
\[
\limsup_{n \to \infty} n \left( \frac{1 + a_{n+1}}{a_n} - 1 \right) \geq 1。
\]
\begin{proof}
假设 $\limsup_{n \to \infty} n \left( \frac{1 + a_{n+1}}{a_n} - 1 \right) < 1$,则存在 $n_0 \in \mathbb{N}$,当 $n \geq n_0$ 时,总有:
\[
n \left( \frac{1 + a_{n+1}}{a_n} - 1 \right) < 1,
\]
即:
\[
\frac{1 + a_{n+1}}{a_n} < 1 + \frac{1}{n},
\]
从而:
\[
a_n - a_{n+1} > \frac{a_n}{n}。
\]

依次可得:
\[
\frac{a_{n_0}}{n_0} - \frac{a_{n_0+1}}{n_0+1} > \frac{a_{n_0}}{n_0} - \frac{a_{n_0+1}}{n_0+1} + \frac{a_{n_0+1}}{n_0+2} > \cdots,
\]
相加可得:
\[
\frac{a_{n_0}}{n_0} > \frac{1}{n_0} + \frac{1}{n_0+1} + \cdots + \frac{1}{n}。
\]

当 $n_0$ 选定后,$\frac{a_{n_0}}{n_0}$ 为定值,而显然:
\[
\frac{1}{n_0} + \cdots + \frac{1}{n}
\]
为无界的,这与假设矛盾。

故假设不成立,得:
\[
\limsup_{n \to \infty} n \left( \frac{1 + a_{n+1}}{a_n} - 1 \right) \geq 1。
\]
\end{proof}

\subsubsection*{1.3.9}
设 $\{a_n\}_{n=1}^\infty$ 为有界数列,记
\[
A_n = \frac{a_1 + a_2 + \cdots + a_n}{n}。
\]
指出 $\limsup a_n$ 与 $\limsup A_n$ 的大小关系并给出证明。

\begin{proof}
$\limsup A_n \leq \limsup a_n$. 下证。

已知 $\{a_n\}$ 有界,故 $\exists M$ 使得 $\lvert a_n \rvert \leq M$,不妨令 $A = \limsup a_n$,$B = \limsup A_n$。

对 $\forall \varepsilon > 0$,由 $A = \limsup A_n$ 及 $\lvert a_n \rvert \leq M$ 可知,在 $\{a_n\}$ 中,只有有限项大于 $A + \varepsilon$,记这些项的最大下标为 $m$。则当 $n > m$ 时,
\[
A_n \leq \frac{mM + (n-m)(A + \varepsilon)}{n} \leq M \frac{m}{n} + A + \varepsilon。
\]
当 $n$ 充分大时,有 $A_n \leq A + \varepsilon$,由 $\varepsilon$ 的任意性可知 $A_n \leq A$,故 $B \leq A$。

因此 $\limsup A_n \leq \limsup a_n$。
\end{proof}

\subsubsection*{1.3.10}
设 $\{a_n\}_{n=1}^\infty$ 是闭区间 $[a,b]$ 内的数列,$f$ 是 $[a,b]$ 上的连续函数。是否一定有
\[
f(\limsup a_n) = \limsup f(a_n)?
\]
说明理由。
\begin{proof}
不一定成立。

不妨令 $[a, b] = [0, 1]$,令 $f$ 是 $[0, 1]$ 上单调连续的概率空间 $P$。

设 $\{a_n\}$ 构成 $P$ 上的一个事件列。那么
\[
P(\limsup A_n) = P\left(\bigcap_{k=1}^\infty \bigcup_{n=k}^\infty A_n\right) = \lim_{k \to \infty} P\left(\bigcup_{n=k}^\infty A_n\right) \geq \limsup_{k \to \infty} P(A_n) = \limsup P(A_n)。
\]
即 $P(\limsup A_n) \geq \limsup P(A_n)$。因此,不一定有 $f(\limsup a_n) = \limsup f(a_n)$。
\end{proof}

\subsection{连续函数作业题}

\subsubsection*{1.4.1}
设 
\[
\lim_{x \to 0} \frac{f(x)}{x} \in \mathbb{R},
\]
又有常数 $\alpha \neq 1$ 使得
\[
f(x) - f(\alpha x) = o(x) \quad (x \to 0).
\]
试证:
\[
f(x) = o(x) \quad (x \to 0).
\]

\begin{proof}
不妨令 $\lim_{x \to 0} \frac{f(x)}{x} = A \in \mathbb{R}$,则显然有
\[
\lim_{x \to 0} \frac{f(\alpha x)}{x} = \alpha A。
\]
结合已知条件可得
\[
0 = \lim_{x \to 0} \frac{f(x) - f(\alpha x)}{x} = A - \alpha A = (1-\alpha)A,
\]
于是显然 $A = 0$。

由此得证 $f(x) = o(x) \quad (x \to 0)$。
\end{proof}

\subsubsection*{1.4.2}
设函数 $f, g$ 在区间 $I$ 上连续,记
\[
F(x) = \min \{f(x), g(x)\}, \quad G(x) = \max \{f(x), g(x)\},
\]
试证:$F, G$ 在 $I$ 上连续。

\begin{proof}
易知
\[
F(x) = \frac{1}{2} \left[ f(x) + g(x) - |f(x) - g(x)| \right], \quad 
G(x) = \frac{1}{2} \left[ f(x) + g(x) + |f(x) - g(x)| \right].
\]
对于任意 $x_0 \in I$,若 $f$ 在点 $x_0$ 连续,$g$ 在点 $x_0$ 连续,则 $f(x), g(x)$ 在 $I$ 上均连续。易知 $|f(x) - g(x)|$ 在 $I$ 上也连续,故 $|f(x) - g(x)|$ 在区间 $I$ 上连续。

故显然,$F(x), G(x)$ 在 $I$ 上均连续。
\end{proof}

\subsubsection*{1.4.3}
设函数 $f$ 在区间 $[a,b]$ 上连续,记
\[
m(x) = \inf_{a \leq t \leq x} f(t).
\]
试证:函数 $m$ 在 $[a,b]$ 上连续。

\begin{proof}
设 $x_0 \in [a, b]$,先证明 $m(x)$ 在点 $x_0$ 右连续。

任给 $\epsilon > 0$,由于 $f(x)$ 在点 $x_0$ 处连续,故存在 $\delta > 0$,使得当 $|x - x_0| < \delta$ 时,恒有 $|f(x) - f(x_0)| < \epsilon$,于是当 $x_0 < x < x_0 + \delta$ 时,有
\[
f(x) > f(x_0) - \epsilon \geq m(x_0) - \epsilon。
\]
而当 $a \leq s \leq x_0$ 时,$f(s) \geq m(x_0) \geq m(x_0) - \epsilon$,由此可知
\[
m(x) \geq \min\{m(x_0), f(x)\} \geq m(x_0) - \epsilon。
\]
又因为 $m(x_0)$ 显然是递减的,故 $m(x_0) > m(x) \geq m(x_0) - \epsilon$(当 $x_0 < x < x_0 + \delta$)。

由此可知
\[
\lim_{x \to x_0^+} m(x) = m(x_0),
\]
即 $m(x)$ 在 $x_0$ 处右连续。

下面证明左连续。不妨设 $f(x)$ 在 $[a, x_0]$ 上的最小值在 $x = x_0$ 时取得,即
\[
m(x_0) = f(x_0) \quad \text{且} \quad m(x) = f(x_0), \quad a \leq x < x_0,
\]
显然成立。

当 $x \to x_0^-$ 时,$m(x) = m(x_0)$,从而右连续。

任给 $\epsilon > 0$,同理,存在 $\delta > 0$,当 $x_0 - \delta < x < x_0$ 时,恒有
\[
f(x) < f(x_0) + \epsilon = m(x_0) + \epsilon。
\]
因此
\[
m(x_0) \leq m(x) \leq m(x_0) + \epsilon,
\]
从而
\[
\lim_{x \to x_0^-} m(x) = m(x_0),
\]
即 $m(x)$ 在 $x_0$ 处左连续。

综上,$m(x)$ 在 $x_0$ 处左右连续,由 $x_0$ 的任意性可知,函数 $m$ 在 $[a, b]$ 上连续。
\end{proof}

\subsubsection*{1.4.4}
设函数 $f$ 在原点的某邻域 $U$ 内有界,且有常数 $\alpha, \beta > 1$ 满足
\[
f(\alpha x) = \beta f(x), \quad x \in U。
\]
试证:$f$ 在原点处连续。

\begin{proof}
在 $f(\alpha x) = \beta f(x)$ 中,令 $x = 0$,而 $\alpha, \beta > 1$,即 $f(0) = f(\alpha \cdot 0) = \beta f(0)$,所以 $f(0) = 0$。

若证明 $\lim_{x \to 0} f(x) = 0$,则由 $U(0, \delta)$ 有界知 $f$ 在原点连续。由 $f(\alpha x) = \beta f(x)$ 推得 $f(\alpha^n x) = \beta^n f(x)$,$f$ 在 $U(0, \delta)$ 内有界,故当 $x \in (-\delta, \delta)$ 时,$\lvert f(x) \rvert < M$。

对任意 $\varepsilon > 0$,由 $\beta > 1$,可知存在 $N \in \mathbb{N}$,使得 $\frac{M}{\beta^n} < \varepsilon$。取定 $n$,当 $\lvert x \rvert < \frac{\delta}{\alpha^n}$ 时,显然有
\[
\lvert f(x) - 0 \rvert = \lvert f(x) \rvert = \beta^n \lvert f(\alpha^n x) \rvert < \beta^n \cdot \frac{M}{\beta^n} < \varepsilon。
\]

故 $\lim_{x \to 0} f(x) = 0 = f(0)$。

综上,$f$ 在原点连续。
\end{proof}

\subsubsection*{1.4.5}
设函数 $f: \mathbb{R} \to \mathbb{R}$ 在 $0,1$ 两点连续,且满足
\[
f(x^2) = f(x), \quad x \in \mathbb{R}.
\]
试证:$f$ 是常函数。

\begin{proof}
当 $x \in (0,1)$,由条件可知 $f(x) = f(x^2) = f(x^4) = \cdots = f(x^{2^n})$。

由于 $f$ 在 $x=0$ 处连续,令 $n \to \infty$,则有
\[
f(x) = \lim_{n \to \infty} f(x^{2^n}) = f(0)。
\]

当 $x = 1$,由 $f$ 的连续性可得
\[
f(1) = \lim_{x \to 1^-} f(x) = \lim_{x \to 1^+} f(x) = f(1)。
\]

又当 $x \in (1,+\infty)$ 时,有
\[
f(x) = f(\sqrt{x}) = f((\sqrt{x})^2) = \cdots = f((x^{\frac{1}{2^n}})^2)。
\]

再由 $f$ 在 $x=1$ 处的连续性有
\[
f(x) = \lim_{n \to \infty} f(x^{\frac{1}{2^n}}) = \lim_{x \to 1^+} f(x) = f(1) = f(0)。
\]

易见 $f$ 为相同常数,故 $f(x) \equiv f(0), \, x \in (-\infty,+\infty)$。

综上,$f$ 为常函数。
\end{proof}

\subsubsection*{1.4.6}
设函数 $f:\mathbb{R} \to \mathbb{R}$ 在 $x=0$ 处有有限的左极限和右极限,且满足
\[
f\left(\frac{x+y}{2}\right) = \frac{f(x) + f(y)}{2}, \quad x, y \in \mathbb{R}.
\]
试证:
\[
f(x) = (f(1) - f(0))x + f(0).
\]

\begin{proof}
由条件有
\[
f\left(\frac{x+y}{2}\right) = \frac{f(x) + f(y)}{2},
\]
记 $b = f(0)$,则有
\[
f(x) + f(y) = f(x+y) + b,
\]
进而有
\[
f(x) - b + f(y) - b = f(x+y) - b。
\]
令 $g(x) = f(x) - b$,则
\[
g(x) + g(y) = g(x+y)。
\]
显然,$g(x)$ 满足 Cauchy 方程,因此 $g(x) = g(1)x$,从而有
\[
f(x) = g(1)x + b = [f(1) - f(0)]x + f(0)。
\]
\end{proof}

\subsubsection*{1.4.7}
设函数 $f$ 在 $[0, 2a]$ 上连续,且 $f(0) = f(2a)$。  
试证:存在 $\xi \in [0, a]$ 使得 $f(\xi) = f(\xi + a)$。
\begin{proof}
因为 $f(x)$ 在 $[0, 2a]$ 上连续,则 $\varphi(x) = f(x) - f(x + a)$ 在 $[0, a]$ 上连续。  
令 $\varphi(0) = f(0) - f(a)$,$\varphi(a) = f(a) - f(2a) = f(a) - f(0)$。

若 $f(0) - f(a) = 0$,则有 $\xi = 0$ 或 $\xi = a$,使得 $f(\xi) = f(\xi + a)$。

若 $f(0) - f(a) \neq 0$,则有 $\varphi(0) \cdot \varphi(a) < 0$,由零点存在定理可知存在 $\xi \in (0, a)$,使得 $\varphi(\xi) = 0$,即 $f(\xi) = f(\xi + a)$。
\end{proof}

\subsubsection*{1.4.8}
设函数 $f$ 在 $[0,1]$ 上连续,且 $f(0) = f(1)$。试证:对于任意正整数 $n$,存在实数 $\xi \in \left[0, 1 - \frac{1}{n}\right]$ 满足
\[
f\left(\xi + \frac{1}{n}\right) = f(\xi)。
\]

\begin{proof}
当 $n=1$ 时,因 $f(0) = f(1)$,则取 $\xi = 0$,结论成立。

当 $n > 1$ 时,令 $F(x) = f\left(x + \frac{1}{n}\right) - f(x)$,则
\[
F(0) + F\left(\frac{1}{n}\right) + \cdots + F\left(\frac{n-1}{n}\right) 
\]
\[
= [f\left(\frac{1}{n}\right) - f(0)] + [f\left(\frac{2}{n}\right) - f\left(\frac{1}{n}\right)] + \cdots + [f(1) - f\left(\frac{n-1}{n}\right)] 
\]
\[
= f(1) - f(0) = 0。
\]

若 $F(0) = F\left(\frac{1}{n}\right) = \cdots = F\left(\frac{n-1}{n}\right) = 0$,取任意 $\xi = 0, \frac{1}{n}, \cdots, \frac{n-1}{n}$ 中一点即可。

若 $F(0), F\left(\frac{1}{n}\right), \cdots, F\left(\frac{n-1}{n}\right)$ 均不全可为 $0$,则有两项极必为异号。

即 $\exists i, j$,使得 $F(\xi_i) F(\xi_j) < 0$,$\xi_i, \xi_j \in \{0, \frac{1}{n}, \cdots, \frac{n-1}{n}\}$。

又因 $f(x)$ 在 $[0,1]$ 连续,由根值存在定理可知,存在 $\xi \in \left[0, 1 - \frac{1}{n}\right]$,使得 $F(\xi) = 0$。

即 $f\left(\xi + \frac{1}{n}\right) = f(\xi)$。

综上,存在 $\xi \in \left[0, 1 - \frac{1}{n}\right]$ 满足 $f\left(\xi + \frac{1}{n}\right) = f(\xi)$。
\end{proof}

\subsubsection*{1.4.9}
设函数 $f, g$ 在 $[a, b]$ 上连续,$f$ 单调,且有数列 $\{x_n\}_{n=1}^\infty \subset [a, b]$ 使得对任意正整数 $n$,有 $g(x_n) = f(x_{n+1})$。试证:存在 $\xi \in [a, b]$ 满足 $f(\xi) = g(\xi)$。
\begin{proof}
设 $F(x) = f(x) - g(x)$,那么 $F(x_n) = f(x_n) - g(x_n) = f(x_n) - f(x_{n+1})$。

根据连续函数的介值性可知,存在 $\xi_n \in [x_n, x_{n+1}]$,使得
\[
F(\xi_n) = \frac{F(x_n)}{n}.
\]

取 $\{ \xi_n \}$ 的一个收敛子列 $\{ \xi_{n_k} \}$,其极限设为 $\xi$,利用 $f$ 的单调收敛性可知
\[
F(\xi) = \lim_{k \to \infty} F(\xi_{n_k}) = \lim_{k \to \infty} \frac{F(x_{n_k})}{n_k} = 0。
\]

所以 $F(\xi) = 0$,即 $f(\xi) = g(\xi)$,$\xi \in [a, b]$。
\end{proof}

\subsubsection*{1.4.10}
设函数 $f$ 在 $[0,+\infty)$ 上连续且有界,又对任意实数 $c$,$f(x) = c$ 只有有限多个解。试证:
\[
\lim_{x \to +\infty} f(x) \text{ 存在}.
\]

\begin{proof}
不妨设 $m_1 < f(x) < M_1, \; x \in [a,+\infty), \; C_1 = \frac{m_1 + M_1}{2}$,因为方程 $f(x) = C_1$ 至多只有有限个实根,所以当 $x$ 充分大时,曲线 $y = f(x)$ 与直线 $y = C_1$ 无公共交点,因此当 $x$ 充分大时,$f(x)$ 始终属于 $(m_1, C_1]$ 或 $[C_1, M_1)$ 内。

若 $f(x)$ 属于前者,则令 $[m_2, M_2] = [m_1, C_1]$,否则令 $[m_2, M_2] = [C_1, M_1]$。依此类推,则可得到闭区间列 $\{[m_n, M_n]\}$,满足:
\begin{enumerate}
    \item $[M_{n+1}, M_{n+1}] \subseteq [M_n, M_n], \; n = 1, 2, 3, \ldots$
    \item $M_n - m_n = \frac{M_1 - m_1}{2^n} \to 0, \; n \to \infty$
    \item 对于每个 $n$,当 $x$ 充分大时,$f(x) \in (m_n, M_n)$
\end{enumerate}

由闭区间套定理,存在唯一的点 $\xi \in \bigcap_{n=1}^{\infty} [m_n, M_n]$,下证 $\lim_{x \to \infty} f(x) = \xi$。

对于任意 $\epsilon > 0$,存在 $N$,当 $n > N$ 时,$[m_n, M_n] \subset (\xi - \epsilon, \xi + \epsilon)$。取 $n_0 > N$,存在 $X > 0$,当 $x > X$ 时,$f(x) \in (m_{n_0}, M_{n_0})$,于是当 $x > X$ 时,有 $|f(x) - \xi| < M_{n_0} - m_{n_0} < 2\epsilon$。

$\epsilon$ 任意,因此 $\lim_{x \to \infty} f(x) = \xi$ 即 $\lim_{x \to \infty} f(x)$ 存在。
\end{proof}

\subsection{可微函数作业题}
\subsubsection*{1.5.1}
设 $f(x)$ 在 $[a, b]$ 上连续,在 $(a, b)$ 内可导,试证:存在点 $\xi \in (a, b)$ 使得
\[
f(\xi) - f(a) = f'(\xi)(b - \xi)。
\]
\begin{proof}
要证 $f(\xi) - f(a) = f'(\xi)(b - \xi)$,只需证 
\[
(b - \xi)f'(\xi) - [f(\xi) - f(a)] = 0
\] 
即可。

构造函数 $F(x) = (b - x)[f(x) - f(a)]$,由题设可知,显然 $F(x)$ 在 $[a, b]$ 上连续,在 $(a, b)$ 上可导。

观察函数 $F(x)$ 可知,$F(a) = F(b) = 0$,因此 $F(x)$ 在 $[a, b]$ 上满足罗尔定理,由罗尔定理可知,存在一点 $\xi \in (a, b)$,满足 $F'(\xi) = 0$。

由 $F'(x) = (b - x)f'(x) - [f(x) - f(a)]$ 可知
\[
F'(\xi) = (b - \xi)f'(\xi) - [f(\xi) - f(a)] = 0 \iff f(\xi) - f(a) = f'(\xi)(b - \xi)。
\]
\end{proof}

\subsubsection*{1.5.3}
设 $f(x), g(x)$ 在 $[a, b]$ 上连续,在 $(a, b)$ 内可导,$g(x) \neq 0$,且有 $f(a)g(b) = f(b)g(a)$。试证:存在点 $\xi \in (a, b)$ 使得
\[
f'(\xi)g(\xi) = f(\xi)g'(\xi)。
\]
\begin{proof}
构造 $F(x) = \frac{f(x)}{g(x)}$,由题设可知,$F(x)$ 在 $[a, b]$ 上连续,在 $(a, b)$ 上可导,且 $F(a) = F(b) = 0$。

显然 $F(x)$ 在 $[a, b]$ 上满足罗尔定理,故至少存在一点 $\xi \in (a, b)$,满足 $F'(\xi) = 0$。

计算 $F'(x)$:
\[
F'(x) = \frac{f'(x)g(x) - f(x)g'(x)}{g^2(x)}
\]
因此
\[
\frac{f'(\xi)g(\xi) - f(\xi)g'(\xi)}{g^2(\xi)} = 0 \quad \text{且} \quad g(\xi) \neq 0,
\]
故
\[
f'(\xi)g(\xi) = f(\xi)g'(\xi)。
\]
\end{proof}

\subsubsection*{1.5.4}
设 $f(x)$ 在 $[0, 3]$ 上连续,在 $(0, 3)$ 内可导,且
\[
f(0) + f(1) + f(2) = 3, \quad f(3) = 1,
\]
试证:存在点 $\xi \in (0,3)$ 使得 $f'(\xi) = 0$。
\begin{proof}
$f(x)$ 在 $[0, 3]$ 上连续,所以 $f(x)$ 在 $[0, 2]$ 上也连续,从而 $f(x)$ 在 $[0, 2]$ 上可取到最大值 $M$ 和最小值 $m$,于是
\[
m \leq f(0) \leq m, \quad m \leq f(1) \leq m, \quad m \leq f(2) \leq M \implies m \leq \frac{f(0) + f(1) + f(2)}{3} \leq M.
\]
由介值定理可知,至少存在一点 $\zeta \in [0, 2]$,使得
\[
f(\zeta) = \frac{f(0) + f(1) + f(2)}{3} = 1.
\]
又因为 $f(x)$ 在 $[\zeta, 3] \subset [0, 3]$ 上连续,在 $(\zeta, 3)$ 上可导,且 $f(\zeta) = 1 = f(3)$,由罗尔定理可知,必存在一点 $\xi \in (\zeta, 3) \subset (0, 3)$,使得
\[
f'(\xi) = 0.
\]
\end{proof}

\subsubsection*{1.5.5}
设 $x_1 = 14$, $x_{n+1} = \sqrt{2 + x_n} \ (n = 1, 2, 3, \cdots)$。

(1) 求 $\lim_{n \to \infty} x_n$;

(2) 求 $\lim_{n \to \infty} \left( \frac{4(x_{n+1} - 2)}{x_n - 2} \right)^{\frac{1}{x_n - 2}}$。
\begin{proof}

\textbf{解5-1:} 不难发现对于 $n \in \mathbb{Z}^+$,都有 $x_n > 0$,首先通过做差法考虑数列的单调性:
\[
x_{n+1} - x_n = \sqrt{2 + x_n} - \sqrt{2 + x_{n-1}} = \frac{x_n - x_{n-1}}{\sqrt{2 + x_n} + \sqrt{2 + x_{n-1}}}.
\]

从第一个等式可以看出 $x_{n+1} - x_n$ 的正负性与 $x_n - x_{n-1}$ 有关,递推可知 $x_{n+1} - x_n \leq x_2 - x_1$,由 $x_2 - x_1 = \sqrt{2 + 14} - 14 = -10 < 0$。

因此显然 $x_n$ 单调递减,结合第二个等式显然可得知 $x_n$ 有下界 2。由单调有界定理,不动点定理和压缩映射定理可知 $\sqrt{2 + A} = A$,显然 $A = 2$ (此数列的非负性),因此:
\[
\lim_{n \to \infty} x_n = A = 2。
\]

\textbf{解5-2:} 结合 5-1 结论,令 $I = \lim_{n \to \infty} \left( \frac{4(x_{n+1} - 2)}{x_n - 2} \right)^{\frac{1}{x_n - 2}}$,做出变量替换 $x_n - 2 = t$,$t \to 0$。

\[
I = \lim_{t \to 0} \left( \frac{4(\sqrt{t+4} - 2)}{t} \right)^{\frac{1}{t}} = \lim_{t \to 0} \left( 1 - t + 4(\sqrt{t + 4} - 2) \right)^{\frac{-t}{t + 4(\sqrt{t + 4} - 2)}}.
\]

利用洛必达法则得:
\[
\lim_{t \to 0} \frac{-t}{t + 4(\sqrt{t + 4} - 2)} = \lim_{t \to 0} \frac{-1}{1 + \frac{4}{2\sqrt{t + 4}}} = \frac{-1}{4}。
\]

故:
\[
I = e^{-\frac{1}{16}}。
\]
\end{proof}

\subsubsection*{1.5.6}
设 $f(x)$ 在 $[a,b]$ 上有二阶导数,且满足 $f'(a) = f'(b) = 0$。试证:存在点 $\xi \in (a,b)$ 使得
\[
|f''(\xi)| \geq 4 \frac{|f(b) - f(a)|}{(b-a)^2}.
\]
\begin{proof}
由泰勒定理,对 $\forall x \in (a,b)$,均存在对应的 $\xi_1, \xi_2 \in (a,b)$,使得
\[
f(x) = 
\begin{cases}
f(a) + f'(a)(x-a) + \frac{f''(\xi_1)}{2!}(x-a)^2, \\
f(b) + f'(b)(x-b) + \frac{f''(\xi_2)}{2!}(x-b)^2
\end{cases}
\]
结合 $f'(a) = f'(b) = 0$,上面两式相减有
\[
|f(b) - f(a)| = \left| \frac{1}{2} f''(\xi_1)(x-a)^2 - \frac{1}{2}f''(\xi_2)(x-b)^2 \right|.
\]
观察题证等式,令 $x = \frac{a+b}{2}$,此时 $(x-a)^2 = (x-b)^2 = \frac{(b-a)^2}{4}$,带入上述表达式可得
\[
|f(b) - f(a)| = \frac{1}{8}(b-a)^2 \cdot |f''(\xi_1) - f''(\xi_2)|.
\]
令 $|f''(\xi)| = \max\{|f''(\xi_1)|, |f''(\xi_2)|\}$,则
\[
|f(b) - f(a)| \leq \frac{1}{8}(b-a)^2 \cdot 2|f''(\xi)|.
\]
因此有
\[
|f(b) - f(a)| \leq \frac{1}{4}(b-a)^2 |f''(\xi)|,
\]
即
\[
|f''(\xi)| \geq 4 \frac{|f(b) - f(a)|}{(b-a)^2}.
\]
\end{proof}

\subsubsection*{1.5.7}
设 $f(x)$ 有 $n+1$ 阶导数,满足
\[
f(a+h) = f(a) + f'(a)h + \frac{f''(a)}{2!}h^2 + \cdots + \frac{f^{(n)}(a+\theta h)}{n!}h^n,
\]
其中 $0 < \theta < 1$,且 $f^{(n+1)}(a) \neq 0$。试证:
\[
\lim_{h \to 0} \theta = \frac{1}{n+1}。
\]

\begin{proof}
已知 $f(x)$ 是区间上 $n+1$ 阶可导函数,所以 $f(a+h)$ 本可以展开到 $h^{n+1}$,即有
\[
f(a+h) = f(a) + f'(a)h + \frac{f''(a)}{2!}h^2 + \cdots + \frac{f^{(n)}(a)}{n!}h^n + \frac{f^{(n+1)}(\eta)}{(n+1)!}h^{n+1},
\]
其中 $\eta \in (a, b)$。

上面两式作差化简后得
\[
\frac{f^{(n+1)}(\eta)}{(n+1)}h = f^{(n)}(a + \theta h) - f^{(n)}(a)。
\]

拉格朗日中值定理得存在 $\xi \in (a, a + h)$,使得
\[
f^{(n)}(a + \theta h) - f^{(n)}(a) = f^{(n+1)}(\xi) \cdot \theta h,
\]
带入得
\[
\frac{f^{(n+1)}(\eta)}{(n+1)} = f^{(n+1)}(\xi) \cdot \theta。
\]

现在利用 $f^{(n+1)}(x)$ 的连续性,结合 $\xi, \eta \to a$ ($h \to 0$),对上述两端关于 $h \to 0$ 取极限有
\[
\frac{f^{(n+1)}(a)}{(n+1)} = f^{(n+1)}(a) \cdot \lim_{h \to 0} \theta。
\]

再结合 $f^{(n+1)}(a) \neq 0$ 可得证
\[
\lim_{h \to 0} \theta = \frac{1}{n+1}。
\]
\end{proof}

\subsubsection*{1.5.8}
设
\[
y = x^3 \ln(1+x),
\]
求 $y^{(99)}(0)$。

\begin{proof}
观察变量形式,根据麦克劳林展开式可得
\[
y = x^3 \cdot \sum_{n=1}^\infty (-1)^{n-1} \frac{x^n}{n} = \sum_{n=1}^\infty (-1)^{n-1} \frac{x^{n+3}}{n}.
\]

又由 Taylor 展开式
\[
y = y(0) + y'(0)x + \frac{y''(0)}{2}x^2 + \cdots + \frac{y^{(99)}(0)}{99!}x^{99} + \cdots
\]

故而由其唯一性可知,有 $n+3=99$,即 $n=96$,此时
\[
y^{(99)}(0) = (-1)^{96-1} \frac{99!}{96} \implies y^{(99)}(0) = -\frac{99!}{96}.
\]
\end{proof}

\subsubsection*{1.5.9}
试比较 $2023^{2024}$ 和 $2024^{2023}$ 的大小。

\begin{proof}
观察发现 $2024 > 2023 > e$,构造函数 $f(x) = \frac{\ln x}{x}$,观察 $f(x)$ 在 $(e, +\infty)$ 上的单调性。

\[
f'(x) = \frac{(\ln x)' \cdot x - x' \cdot \ln x}{x^2} = \frac{1 - \ln x}{x^2} < 0, \quad x \in (e, +\infty)
\]

显然,$f(x)$ 在 $(e, +\infty)$ 上单调递减,即 $f(2024) < f(2023)$。

\[
0 < \frac{\ln 2024}{2024} < \frac{\ln 2023}{2023} \implies 2023 \ln 2024 < 2024 \ln 2023 \implies \ln 2024^{2023} < \ln 2023^{2024}
\]

显然
\[
2024^{2023} < 2023^{2024}.
\]
\end{proof}

\subsubsection*{1.5.10}
就 $k$ 的不同取值情况,讨论方程
\[
x - \frac{\pi}{2} \sin x = k
\]
在区间 $\left(0, \frac{\pi}{2}\right)$ 内根的个数。

\begin{proof}
观察方程形式,构造函数
\[
f(x) = x - \frac{\pi}{2} \sin x,
\]
研究 $f(x)$ 在 $\left(0, \frac{\pi}{2}\right)$ 上的单调性。

计算 $f'(x)$,得
\[
f'(x) = 1 - \frac{\pi}{2} \cos x.
\]
令 $f'(x) = 0$,可得驻点
\[
x_0 = \arccos \frac{2}{\pi}.
\]
结合 $f(0) = f\left(\frac{\pi}{2}\right) = 0$,研究 $y = f(x)$ 与 $y = k$ 的交点。

分情况讨论:
\begin{itemize}
    \item[i.] 当 $k \geq 0$ 或 $k < f\left(\arccos \frac{2}{\pi}\right)$ 时,原方程没有根;
    \item[ii.] 当 $k = f\left(\arccos \frac{2}{\pi}\right)$ 时,原方程有一个根;
    \item[iii.] 当 $f\left(\arccos \frac{2}{\pi}\right) < k < 0$ 时,原方程有两个根。
\end{itemize}
\end{proof}

\subsubsection*{1.5.11}
设 $p(x)$ 是一个多项式,试证:
\begin{enumerate}
    \item $\lim_{x \to \infty} \frac{p(x)}{e^{|x|}} = 0$;
    \item $\lim_{x \to \infty} \frac{p(x)}{e^{x^2}} = 0$.
\end{enumerate}

\begin{proof}
解11-1:不妨设 $p(x) = a_0 + a_1 x + \cdots + a_n x^n \, (n \in \mathbb{Z}^+, a_n \in \mathbb{R})$,我们有 $p^{(n)}(x) = n! a_n$。

先考虑 $x > 0$ 的情况,此时
\[
\lim_{x \to \infty} \frac{p(x)}{e^{|x|}} = \lim_{x \to \infty} \frac{p(x)}{e^x},
\]
此时 $p(x), e^x \to \infty \, (x \to +\infty)$,洛必达法则
\[
\lim_{x \to \infty} \frac{p(x)}{e^x} = \lim_{x \to \infty} \frac{p^{(n)}(x)}{(e^x)^{(n)}} = \lim_{x \to \infty} \frac{n! a_n}{e^x} = 0。
\]

同理 $x < 0$ 的情况,$p(x), e^{|x|} \to \infty \, (x \to -\infty)$,显然可以得到相同的结论。

因此,综上所述
\[
\lim_{x \to \infty} \frac{p(x)}{e^{|x|}} = 0。
\]

解11-2:参考 11-1 的解法,
\[
\lim_{x \to \infty} \frac{p(x)}{e^{x^2}} = 0,
\]
分子有限次求导后结果为 $0$,而分母任意阶导数在 $x \to \infty$ 时都是无穷大量,反复使用洛必达法则即可得
\[
\lim_{x \to \infty} \frac{p(x)}{e^{x^2}} = 0。
\]
\end{proof}

\subsubsection*{1.5.12}
定义函数
\[
f(x) =
\begin{cases}
e^{-x^{-2}}, & \text{当 } x \neq 0; \\
0, & \text{当 } x = 0.
\end{cases}
\]

\begin{enumerate}
\item 对任意正整数 $n$,总是有多项式 $p_n(x)$ 使得当 $x \neq 0$ 时
\[
f^{(n)}(x) = p_n\left(\frac{1}{x}\right)e^{-x^{-2}};
\]
\item $f^{(n)}(0) = 0, \quad n = 0, 1, 2, \dots.$
\end{enumerate}

\begin{proof}
\textbf{解12-1: 使用数学归纳法}
\begin{enumerate}
    \item 当 $n=1$ 时,$f'(x) = \frac{2}{x^3}e^{-x^{-2}}$ 可知结论成立。
    \item 现在设 $f^{(k)}(x) = p_k\left(\frac{1}{x}\right)e^{-x^{-2}}$ 成立,其中 $p_k\left(\frac{1}{x}\right)$ 是 $\frac{1}{x}$ 的多项式。
    \item 则
    \[
    f^{(k+1)}(x) = p_k'\left(\frac{1}{x}\right)\left(-\frac{1}{x^2}\right)e^{-x^{-2}} + p_k\left(\frac{1}{x}\right)\left(\frac{2}{x^3}\right)e^{-x^{-2}} = p_{k+1}\left(\frac{1}{x}\right)e^{-x^{-2}},
    \]
    其中
    \[
    p_{k+1}\left(\frac{1}{x}\right) = \frac{2}{x^3}p_k\left(\frac{1}{x}\right) - \frac{1}{x^2}p_k'\left(\frac{1}{x}\right)
    \]
    仍是 $\frac{1}{x}$ 的多项式。因此假设成立,故对任意正整数 $n$,总是有多项式 $p_n(x)$ 使得当 $x \neq 0$ 时有
    \[
    f^{(n)}(x) = p_n\left(\frac{1}{x}\right)e^{-x^{-2}}。
    \]
\end{enumerate}

\textbf{解12-2:}
结合 11-2 和 12-1 两个结论,再用数学归纳法证明本题:
\begin{enumerate}
    \item 当 $n=1$ 时,有
    \[
    f'(0) = \lim_{x \to 0} \frac{f(x) - f(0)}{x} = \lim_{x \to 0} \frac{e^{-x^{-2}} - 0}{x} \overset{y = \frac{1}{x}}{=} \lim_{y \to +\infty} \frac{e^{-y^2}}{\frac{1}{y}} = \lim_{y \to +\infty} \frac{y}{e^{y^2}} = 0。
    \]
    \item 现在设 $f^{(k)}(0) = 0$ 成立。
    \item 则
    \[
    f^{(k+1)}(0) = \lim_{x \to 0} \frac{f^{(k)}(x) - f^{(k)}(0)}{x} = \lim_{x \to 0} \frac{p_k\left(\frac{1}{x}\right)e^{-x^{-2}}}{x} \overset{y = \frac{1}{x}}{=} \lim_{y \to +\infty} \frac{p_k(y)e^{-y^2}}{\frac{1}{y}} = 0。
    \]
\end{enumerate}

综上所述,
\[
f^{(n)}(0) = 0 \quad (n = 1, 2, 3, \dots)。
\]
\end{proof}

\subsubsection*{1.5.13}

设 $f(x)$ 在全实轴上可导,且有实数 $A$ 使得
\[
\lim_{x \to +\infty} \left( f(x) + x f'(x) \right) = A.
\]
试证:
\[
\lim_{x \to +\infty} f(x) = A.
\]

\begin{proof}
设函数 $f$ 和 $g$ 在 $(a, b)$ 上可导,$g(x) \neq 0$ 且 $\lim_{x \to a^+} g(x) = \infty$。如果极限
\[
\lim_{x \to a^+} \frac{f'(x)}{g'(x)}
\]
存在,那么
\[
\lim_{x \to a^+} \frac{f'(x)}{g'(x)} = \lim_{x \to a^+} \frac{f(x)}{g(x)}.
\]

证明如下:

令
\[
\lim_{x \to a^+} \frac{f'(x)}{g'(x)} = l \in \mathbb{R},
\]
对任意给定的 $\varepsilon > 0$,存在一个 $\delta > 0$,当 $x \in (a, a + \delta)$ 时,有
\[
l - \varepsilon < \frac{f'(x)}{g'(x)} < l + \varepsilon.
\]
因此对 $(x, c) \subset (a, a + \delta)$,根据 Cauchy 中值定理,必存在 $\xi \in (x, c)$,使得
\[
l - \varepsilon < \frac{f(x) - f(c)}{g(x) - g(c)} = \frac{f'(\xi)}{g'(\xi)} < l + \varepsilon.
\]
固定 $c$,对 $x \to a^+$ 取上极限,得
\[
\limsup_{x \to a^+} \frac{f(x)}{g(x)} \leq l + \varepsilon.
\]

令 $\varepsilon \to 0$,得
\[
\limsup_{x \to a^+} \frac{f(x)}{g(x)} \leq l.
\]

同理可得
\[
\liminf_{x \to a^+} \frac{f(x)}{g(x)} \geq l.
\]

因此
\[
\lim_{x \to a^+} \frac{f(x)}{g(x)} = l \implies \lim_{x \to a^+} \frac{f'(x)}{g'(x)} = \lim_{x \to a^+} \frac{f(x)}{g(x)} = l.
\]

令 $h(x) = x f(x), g(x) = x$,由题目可知,$h'(x), g(x), f(x), h(x)$ 在 $\mathbb{R}$ 上均可导。

分别令 $a = \pm \infty$,即可得到
\[
\frac{h'(x)}{g'(x)} = \lim_{x \to \infty} \frac{h(x)}{g(x)} = \lim_{x \to \infty} f(x) = A.
\]

因此
\[
\lim_{x \to \infty} f(x) = A.
\]
\end{proof}

\subsubsection*{1.5.14}

设 $a = \sqrt[3]{3}$,$x_1 = a$,$x_{n+1} = a^{x_n} \, (n = 1, 2, \cdots)$。

试证:
\begin{enumerate}
    \item $\lim_{n \to \infty} x_n$ 存在;
    \item $\lim_{n \to \infty} x_n \neq 3$。
\end{enumerate}

\begin{proof}
\textbf{解14-(1)}首先用数学归纳法证明对 $\forall n \in \mathbb{Z}^+$,都有 $\sqrt[3]{3} \leq x_n < 3$。

\begin{enumerate}
    \item 对 $n = 1, 2$ 显然有 $\sqrt[3]{3} \leq x_1 < x_2 < 3$;
    \item 现在设当 $k = n$ 时,$\sqrt[3]{3} \leq x_n < 3$ 仍成立;
    \item 当 $k = n + 1$ 时,显然 $\sqrt[3]{3} \leq x_n < x_{n+1} = a^{x_n} = a^{a^n} < 3$ 成立。
\end{enumerate}

因此,对 $\forall n \in \mathbb{Z}^+$,都有 $\sqrt[3]{3} \leq x_n < 3$,从而得到 $x_n$ 有界。

然后通过做差法研究 $x_n$ 的单调性:

\[
x_{n+2} - x_{n+1} = a^{x_{n+1}} - a^{x_n} = a^{x_n} \left( a^{x_{n+1} - x_n} - 1 \right) = a^{x_n} \left( a^{x_{n+1} - x_n} - 1 \right), \quad [a^n > 0, a^0 = 1].
\]

不难看出 $x_{n+2} - x_{n+1}$ 的正负性与 $x_{n+1} - x_n$ 相同,反复递代可知只与 $x_2 - x_1$ 的正负性有关。

$x_2 - x_1 > 0$,因此对 $\forall n \in \mathbb{Z}^+$,都有 $x_{n+1} - x_n > 0$,易知 $x_n$ 单调递增。

由单调有界定理可知,数列 $x_n$ 必有极限,即
\[
\lim_{n \to \infty} x_n \text{ 存在}.
\]

\textbf{解14-(2)}
首先构造函数 $h(x) = \frac{\ln x}{x}$,研究 $h(x)$ 的单调性。$h'(x) = \frac{1 - \ln x}{x^2} = 0$($x = e$)$h(x)$ 在 $(0, e)$ 上单调递增,在 $(e, +\infty)$ 上单调递减。在 $(0, +\infty)$ 上有极大值 $\frac{1}{e}$。

再构造函数 $p(x) = x^x$,不难看出 $p(x) = \frac{1}{e} e^{x \ln x} = e^{h(x)}$,因此有 $p(3) < p(e) \implies 3^{\frac{1}{e}} < \frac{1}{e}$。

考虑方程 $f(x) = x^x$ 的不动点问题,令 $g(x) = x - a^x$,显然 $g(x) = 0$ 的解就是 $f(x)$ 的不动点。

$g'(x) = a^x \ln a - 1, g''(x) = a^x (\ln a)^2 > 0$,显然 $g(x)$ 单调递增。当 $g'(x) < 0$ 时,$x^* = -\frac{\ln a}{\ln a}$。

因此 $g(x)$ 有最小值 $g(x) = \frac{1 + \ln \ln a}{\ln a}$,当 $g'(x) < 0$ 时,解得 $x^* = \frac{1}{e^e}$。

当 $1 < a = \frac{1}{33}$ 时,由于 $g(x) = 0$ 是凸函数,$g(x) = 0$ 有且仅有两个根,分别令其为 $x_1, x_2$,它们是 $f(x)$ 的两个不动点,不失一般性,令 $x_1 < x^* < x_2$。

\begin{enumerate}
    \item 对于 $x_2 = x^* = -\frac{\ln \ln a}{\ln a}$,由于 $a^{x^2} = x_2$,可得 $|f'(x_2)| = |a^{x^2} \ln a| = x_2 \ln a > -\ln a > 1$。
    \item 对于 $x_1 < x^*$,由于 $f'(x) = x^{x_1} \ln a$,可得 $|f'(x_1)| = |a^{x_1} \ln a| = f(x) \ln a < 1$。
\end{enumerate}

因此,当 $1 < a < e^{e}$ 时,$f(x)$ 有且仅有两个不动点,且较小的点为稳定的不动点,较大的点为不稳定的不动点。

对于 $g(x) = 0$,显然有一个解 $x = 3$,但
\[
3 - x = 3 + \frac{\ln a}{\ln \ln 3} = 3 + \frac{\ln (3 \ln 3)}{\ln 3} = 3 (\ln 3 \ln \ln 3).
\]

显然 $x_2 = 3$ 是较大的不稳定的不动点,故 $f(x)$ 不可能收敛于 $x_2 = 3$。因此,可得
\[
\lim_{n \to \infty} x_n \neq 3.
\]
\end{proof}

\subsection{函数性质作业题}
\subsubsection*{1.6.1}
设 $n$ 为正整数,函数
\[
f(x) =
\begin{cases} 
x^n \sin \ln |x|, & x \neq 0; \\
0, & x = 0.
\end{cases}
\]

试证:$f$ 在 $x = 0$ 处有 $n-1$ 阶导数,但 $f^{(n)}(0)$ 不存在。

\begin{proof}
当 $n = 1$ 时,由于
\[
\lim_{x \to 0} \frac{f(x) - f(0)}{x} = \lim_{x \to 0} \frac{x^2 \sin (\ln |x|) - 0}{x} = \lim_{x \to 0} x \sin (\ln |x|) = 0,
\]
所以 $f'(0) = 0$。

又因为
\[
f'(x) =
\begin{cases}
2x \sin (\ln |x|) + x \cos (\ln x), & x > 0, \\
2x \sin (\ln |x|) + x \cos (\ln (-x)), & x < 0,
\end{cases}
\]
所以
\[
f''(x) =
\begin{cases}
x \big( 2 \sin (\ln |x|) + \cos (\ln |x|) \big), & x \neq 0, \\
0, & x = 0.
\end{cases}
\]

由于
\[
\lim_{x \to 0} \frac{f'(x) - f'(0)}{x} = \lim_{x \to 0} \frac{x \big( 2 \sin (\ln |x|) + \cos (\ln |x|) \big) - 0}{x} 
= \lim_{x \to 0} \big( 2 \sin (\ln |x|) + \cos (\ln |x|) \big),
\]
不存在,因此 $f''(0)$ 不存在。

假设 $n = k$ 时,结论成立。当 $n = k + 1$ 时,可得
\[
f'(x) =
\begin{cases}
x^{k+1} \big( (k + 2) \sin (\ln |x|) + \cos (\ln |x|) \big), & x \neq 0, \\
0, & x = 0.
\end{cases}
\]

类似证明,结论成立。

由此可知,当 $f$ 在 $x = 0$ 处有 $n - 1$ 阶导数,必有 $f^{(n)}(0)$ 不存在。
\end{proof}

\subsubsection*{1.6.2}

试构造可微函数 $f(x)$ 使得 $f'(0) > 0$,但对任意 $\delta > 0$,$f(x)$ 在开区间 $(-\delta, \delta)$ 上都不是递增函数。

\begin{proof}
设
\[
f(x) =
\begin{cases}
\frac{x}{2} + x^2 \sin \frac{1}{x}, & x \neq 0; \\
0, & x = 0.
\end{cases}
\]

计算导数,首先验证 $x = 0$ 时:
\[
f'(0) = \lim_{x \to 0} \frac{f(x) - f(0)}{x} = \lim_{x \to 0} \frac{\frac{x}{2} + x^2 \sin \frac{1}{x} - 0}{x}
= \lim_{x \to 0} \left( \frac{1}{2} + x \sin \frac{1}{x} \right) = \frac{1}{2} > 0.
\]

对于 $x \neq 0$,$f'(x)$ 为:
\[
f'(x) = \frac{1}{2} - \cos \frac{1}{x} + 2x \sin \frac{1}{x}.
\]

当 $x_k = \frac{1}{k\pi} \, (k = 1, 2, \dots)$ 时,
\[
f'(x_k) = \frac{1}{2} - (-1)^k.
\]

从上述公式可见,在包含零点的任意开区间内,$f'(x)$ 可以取不同符号的值。因此,$f(x)$ 在该开区间内不是单调函数。
\end{proof}

\subsubsection*{1.6.3}

试构造在整个实数轴上可导的函数 $f(x)$,使得对任意给定的 $\delta > 0$ 和 $\xi \in \mathbb{R}$,总是有一个点列
\[
x_n \in (0, \delta), \quad n = 1, 2, 3, \dots
\]
使得
\[
\lim_{n \to \infty} x_n = 0, \quad \text{并且} \quad f'(x_n) = \xi.
\]
\begin{proof}
定义函数
\[
f(x) =
\begin{cases}
x \sin \left( \frac{1}{x} \right), & x \neq 0, \\
0, & x = 0.
\end{cases}
\]

1. $f(x)$ 在整个实数轴上可导:当 $x \neq 0$ 时,有
\[
f'(x) = \sin \left( \frac{1}{x} \right) - \frac{\cos \left( \frac{1}{x} \right)}{x}.
\]

当 $x = 0$ 时,计算
\[
f'(0) = \lim_{x \to 0} \frac{f(x) - f(0)}{x} = \lim_{x \to 0} \sin \left( \frac{1}{x} \right) = 0.
\]

2. 导数为任意值 $\xi$ 的点列:选择点列 $x_n = \frac{1}{n\pi + \arccos(-\xi)}$,有
\[
f'(x_n) = \sin \left( \frac{1}{x_n} \right) - \frac{\cos \left( \frac{1}{x_n} \right)}{x_n} = \xi.
\]

3. $\lim_{n \to \infty} x_n = 0$:因为 $x_n = \frac{1}{n\pi + \arccos(-\xi)}$,随着 $n \to \infty$,有 $\lim_{n \to \infty} x_n = 0$。

综上,函数 $f(x)$ 满足题意。
\end{proof}

\subsubsection*{1.6.4}

试构造可微函数 $f(x)$ 使得 $f'(0) > 0$,但对任意 $\delta > 0$,$f(x)$ 在开区间 $(-\delta, \delta)$ 上都不是递增函数。

\begin{proof}
设
\[
f(x) =
\begin{cases}
\frac{x}{2} + x^2 \sin \frac{1}{x}, & x \neq 0; \\
0, & x = 0.
\end{cases}
\]

计算导数,首先验证 $x = 0$ 时:
\[
f'(0) = \lim_{x \to 0} \frac{f(x) - f(0)}{x} = \lim_{x \to 0} \frac{\frac{x}{2} + x^2 \sin \frac{1}{x} - 0}{x}
= \lim_{x \to 0} \left( \frac{1}{2} + x \sin \frac{1}{x} \right) = \frac{1}{2} > 0.
\]

对于 $x \neq 0$,$f'(x)$ 为:
\[
f'(x) = \frac{1}{2} - \cos \frac{1}{x} + 2x \sin \frac{1}{x}.
\]

当 $x_k = \frac{1}{k\pi} \, (k = 1, 2, \dots)$ 时,
\[
f'(x_k) = \frac{1}{2} - (-1)^k.
\]

从上述公式可见,在包含零点的任意开区间内,$f'(x)$ 可以取不同符号的值。因此,$f(x)$ 在该开区间内不是单调函数。
\end{proof}

\subsubsection*{1.6.5}

试构造在整个实数轴上有任意阶导数的函数 $f(x)$,使得 $f(x)$ 的任意阶 Maclaurin 多项式都等于零,但对任意 $x \neq 0$,都有
\[
f(x) > 0.
\]
\begin{proof}
从可微函数)第 12 题(1.5.12)可以得到如下结论:

设
\[
f(x) =
\begin{cases}
e^{-x^{-2}}, & x \neq 0; \\
0, & x = 0.
\end{cases}
\]

有
\[
f^{(n)}(0) = 0.
\]

由此结论可知,此函数在 $x = 0$ 这一点的泰勒展开式为零多项式,即此函数的任意阶 Maclaurin 多项式都等于零,且满足题干所要求的值域问题。
\end{proof}

\subsection{积分作业题}
\subsubsection*{1.7.1}

求积分
\[
I = \int_{0}^{2} \frac{x}{e^x + e^{2-x}} \, \mathrm{d}x.
\]

\begin{proof}
设
\[
I = \int_{0}^{2} \frac{x}{e^x + e^{2-x}} \, \mathrm{d}x = I_1 + I_2,
\]
其中
\[
I_1 = \int_{0}^{1} \frac{x}{e^x + e^{2-x}} \, \mathrm{d}x, \quad I_2 = \int_{1}^{2} \frac{x}{e^x + e^{2-x}} \, \mathrm{d}x.
\]

对于 $I_2$,作变量替换 $t = 2 - x$,则有
\[
I_2 = \int_{1}^{2} \frac{2 - x}{e^x + e^{2-x}} \, \mathrm{d}x = \int_{0}^{1} \frac{t}{e^{2-t} + e^t} \, \mathrm{d}t.
\]
注意到 $I_2 = -I_1 + \int_{0}^{1} \frac{2}{e^x + e^{2-x}} \, \mathrm{d}x$。

因此
\[
I = I_1 + I_2 = 2 \int_{0}^{1} \frac{\mathrm{d}x}{e^x + e^{2-x}}.
\]

令 $t = e^x$,则
\[
\int_{0}^{1} \frac{\mathrm{d}x}{e^x + e^{2-x}} = \int_{1}^{e} \frac{\mathrm{d}t}{t^2 + e^2}.
\]
从而
\[
I = \frac{2}{e} \arctan \frac{t}{e} \bigg|_{1}^{e} = \frac{2}{e} \left( \arctan 1 - \arctan \frac{1}{e} \right).
\]

计算得
\[
I = \frac{2}{e} \left( \frac{\pi}{4} - \arctan \frac{1}{e} \right).
\]
\end{proof}

\subsubsection*{1.7.2}

设函数 $f$ 是 $[0, 1]$ 上的连续函数,且满足
\[
\int_{0}^{1} x^n f(x) \, \mathrm{d}x = 1, \quad \int_{0}^{1} x^k f(x) \, \mathrm{d}x = 0, \quad k = 0, 1, \dots, n-1.
\]

试证:
\[
\max_{0 \leq x \leq 1} |f(x)| \geq 2^n (n + 1).
\]

\begin{proof}
此题可化简为证明:存在 $\xi \in [0, 1]$,使得 $|f(\xi)| \geq 2^n(n+1)$。

使用反证法,假设对 $\forall x \in [0, 1]$,$|f(x)| < 2^n(n+1)$,则
\[
I = \int_{0}^{1} x^n f(x) \, \mathrm{d}x = \int_{0}^{1} (x - \frac{1}{2})^n f(x) \, \mathrm{d}x.
\]
根据绝对值不等式,有
\[
\leq \int_{0}^{1} |(x - \frac{1}{2})^n| \cdot |f(x)| \, \mathrm{d}x < 2^n(n+1) \int_{0}^{1} |(x - \frac{1}{2})^n| \, \mathrm{d}x.
\]
\[
= 2^n(n+1) \left[\int_{0}^{1/2} (x - \frac{1}{2})^n \, \mathrm{d}x + \int_{1/2}^{1} (x - \frac{1}{2})^n \, \mathrm{d}x \right].
\]
\[
< 2^n(n+1) \cdot 2 \int_{0}^{1/2} (x - \frac{1}{2})^n \, \mathrm{d}x = 2^{n+1}(n+1) \cdot \frac{1}{n+1} (x - \frac{1}{2})^{n+1} \bigg|_{0}^{1/2} = 1.
\]

矛盾。因此,必定存在 $\xi \in [0, 1]$,使得 $|f(\xi)| \geq 2^n(n+1)$。

从而
\[
\max_{0 \leq x \leq 1} |f(x)| \geq 2^n(n+1).
\]
\end{proof}

\subsubsection*{1.7.3}

设函数 $f$ 在 $[0, 1]$ 上有二阶连续导数,$f(0) = f(1) = 0$,且当 $x \in (0, 1)$ 时 $f(x) \neq 0$。试证:
\[
\int_{0}^{1} \left| \frac{f''(x)}{f(x)} \right| \, \mathrm{d}x \geq 4.
\]

\begin{proof}
利用连续函数性质及拉格朗日微分中值定理证明。

由于题意,设 $f(x)$ 在 $[0, 1]$ 上有最大值,记为 $f(x_0)$,其中 $x_0 \in (0, 1)$。

由拉格朗日微分中值定理,有
\[
\frac{f(x_0) - f(0)}{x_0} = f'(\xi_1), \quad \xi_1 \in (0, x_0);
\]
\[
\frac{f(1) - f(x_0)}{1 - x_0} = f'(\xi_2), \quad \xi_2 \in (x_0, 1).
\]

即
\[
f'(\xi_1) = \frac{f(x_0)}{x_0}, \quad f'(\xi_2) = -\frac{f(x_0)}{1 - x_0}.
\]

于是
\[
\int_{0}^{1} \left| \frac{f''(x)}{f(x)} \right| \, \mathrm{d}x 
\geq \int_{0}^{\xi_1} \left| \frac{f''(x)}{f(x)} \right| \, \mathrm{d}x + \int_{\xi_1}^{\xi_2} \left| \frac{f''(x)}{f(x)} \right| \, \mathrm{d}x + \int_{\xi_2}^{1} \left| \frac{f''(x)}{f(x)} \right| \, \mathrm{d}x.
\]

由拉格朗日中值定理,
\[
\int_{0}^{1} \left| \frac{f''(x)}{f(x)} \right| \, \mathrm{d}x \geq \frac{1}{f(x_0)} \left( \frac{f(x_0)}{x_0} + \frac{f(x_0)}{1 - x_0} \right).
\]

进一步化简得
\[
\int_{0}^{1} \left| \frac{f''(x)}{f(x)} \right| \, \mathrm{d}x \geq \frac{1}{x_0 (1 - x_0)}.
\]

当 $x_0 = \frac{1}{2}$ 时,$x_0(1 - x_0) = \frac{1}{4}$,从而
\[
\int_{0}^{1} \left| \frac{f''(x)}{f(x)} \right| \, \mathrm{d}x \geq 4.
\]

故得证。
\end{proof}

\subsubsection*{1.7.4}

设函数 $f(x)$ 在 $[a, b]$ 上连续可导。试证:
\[
\max_{a \leq x \leq b} |f(x)| \leq \frac{1}{b-a} \left| \int_{a}^{b} f(x) \, \mathrm{d}x \right| + \int_{a}^{b} |f'(x)| \, \mathrm{d}x.
\]
\begin{proof}
由于 $f(x)$ 在 $[a, b]$ 上连续,可设 $|f(\xi)| = \max_{a \leq x \leq b} |f(x)|, \xi \in [a, b]$ 及
$|f(\eta)| = \min_{a \leq x \leq b} |f(x)|, \eta \in [a, b]$。于是
\[
\max_{a \leq x \leq b} |f(x)| - \min_{a \leq x \leq b} |f(x)| = |f(\xi)| - |f(\eta)| \leq |f(\xi) - f(\eta)|
= \left| \int_{\eta}^{\xi} f'(x) \, \mathrm{d}x \right| \leq \int_{a}^{b} |f'(x)| \, \mathrm{d}x.
\]

另一方面,由积分中值定理,$\exists \zeta \in [a, b]$,使得
\[
f(\zeta) = \frac{1}{b-a} \int_{a}^{b} f(x) \, \mathrm{d}x,
\]
于是
\[
\min_{a \leq x \leq b} |f(x)| \leq |f(\zeta)| = \frac{1}{b-a} \left| \int_{a}^{b} f(x) \, \mathrm{d}x \right|.
\]

因此
\[
\max_{a \leq x \leq b} |f(x)| = \min_{a \leq x \leq b} |f(x)| + (\max_{a \leq x \leq b} |f(x)| - \min_{a \leq x \leq b} |f(x)|)
\]
\[
\leq \frac{1}{b-a} \left| \int_{a}^{b} f(x) \, \mathrm{d}x \right| + \int_{a}^{b} |f'(x)| \, \mathrm{d}x.
\]

故得证。
\end{proof}

\subsubsection*{1.7.5}

设函数 $f(x)$ 在 $[0, a]$ 上二阶连续可导 $(a > 0)$,且 $f''(x) \geq 0$。试证:
\[
\int_{0}^{a} f(x) \, \mathrm{d}x \geq a f\left(\frac{a}{2}\right).
\]

\begin{proof}
将 $f(x)$ 在 $x = \frac{a}{2}$ 展开成 1 阶 Taylor 公式,有
\[
f(x) = f\left(\frac{a}{2}\right) + f'\left(\frac{a}{2}\right) \left(x - \frac{a}{2}\right) + \frac{1}{2} f''(\xi) \left(x - \frac{a}{2}\right)^2, \quad (0 < \xi < a).
\]

由 $f''(x) \geq 0$,得到
\[
f(x) \geq f\left(\frac{a}{2}\right) + f'\left(\frac{a}{2}\right) \left(x - \frac{a}{2}\right).
\]

对上述不等式两边从 $0$ 到 $a$ 积分,由于
\[
\int_{0}^{a} \left(x - \frac{a}{2}\right) \, \mathrm{d}x = 0,
\]
就得到
\[
\int_{0}^{a} f(x) \, \mathrm{d}x \geq a f\left(\frac{a}{2}\right).
\]
\end{proof}

\subsubsection*{1.7.6}

设函数 $f$ 在区间 $[-\pi, \pi]$ 上是凸函数,$f'(x)$ 在 $(-\pi, \pi)$ 内存在且有界。试证:
\[
\int_{-\pi}^{\pi} f(x) \cos((2n+1)x) \, \mathrm{d}x \leq 0.
\]
\begin{proof}
由条件可知,$f'(x)$ 在 $[-\pi, \pi]$ 上必为一递增函数。

即有
\[
\int_{-\pi}^{\pi} f(x) \cos((2n+1)x) \, \mathrm{d}x = -\frac{1}{2n+1} \int_{-\pi}^{\pi} f'(x) \sin((2n+1)x) \, \mathrm{d}x \geq 0.
\]
\end{proof}

\subsubsection*{1.7.7}

设函数 $f(x), g(x)$ 在区间 $[a, b]$ 上可积,试证:
\[
\left( \int_{a}^{b} f(x) g(x) \, \mathrm{d}x \right)^2 \leq \int_{a}^{b} f(x)^2 \, \mathrm{d}x \cdot \int_{a}^{b} g(x)^2 \, \mathrm{d}x.
\]

\begin{proof}
设 $t$ 为实参数,则
\[
\int_{a}^{b} \left( f(x) + t g(x) \right)^2 \, \mathrm{d}x \geq 0,
\]
即
\[
\int_{a}^{b} g(x)^2 \, \mathrm{d}x \cdot t^2 + 2 \left[ \int_{a}^{b} f(x) g(x) \, \mathrm{d}x \right] t + \int_{a}^{b} f(x)^2 \, \mathrm{d}x \geq 0.
\]

令
\[
A = \int_{a}^{b} g(x)^2 \, \mathrm{d}x, \quad B = \int_{a}^{b} f(x) g(x) \, \mathrm{d}x, \quad C = \int_{a}^{b} f(x)^2 \, \mathrm{d}x,
\]
作为 $t$ 的一元二次不等式 $A t^2 + 2Bt + C \geq 0$,则必有 $B^2 - AC \leq 0$,即 $B^2 \leq AC$。

因此
\[
\left( \int_{a}^{b} f(x) g(x) \, \mathrm{d}x \right)^2 \leq \int_{a}^{b} f(x)^2 \, \mathrm{d}x \cdot \int_{a}^{b} g(x)^2 \, \mathrm{d}x.
\]
\end{proof}

\subsubsection*{1.7.8}

设函数 $f(x), g(x)$ 在区间 $[a, b]$ 上可积,试证:
\[
\left( \int_{a}^{b} \left( f(x) + g(x) \right)^2 \, \mathrm{d}x \right)^{\frac{1}{2}} \leq 
\left( \int_{a}^{b} f(x)^2 \, \mathrm{d}x \right)^{\frac{1}{2}} + \left( \int_{a}^{b} g(x)^2 \, \mathrm{d}x \right)^{\frac{1}{2}}.
\]
\begin{proof}
由展开式
\[
\left( \left( \int_{a}^{b} f^2(x) \, \mathrm{d}x \right)^{\frac{1}{2}} + \left( \int_{a}^{b} g^2(x) \, \mathrm{d}x \right)^{\frac{1}{2}} \right)^2 
\]
\[
= \int_{a}^{b} f^2(x) \, \mathrm{d}x + \int_{a}^{b} g^2(x) \, \mathrm{d}x + 2 \sqrt{\int_{a}^{b} f^2(x) \, \mathrm{d}x \cdot \int_{a}^{b} g^2(x) \, \mathrm{d}x}.
\]

根据 Cauchy-Schwarz 不等式,
\[
\sqrt{\int_{a}^{b} f^2(x) \, \mathrm{d}x \cdot \int_{a}^{b} g^2(x) \, \mathrm{d}x} \geq \left| \int_{a}^{b} f(x) g(x) \, \mathrm{d}x \right|.
\]

因此
\[
\left( \left( \int_{a}^{b} f^2(x) \, \mathrm{d}x \right)^{\frac{1}{2}} + \left( \int_{a}^{b} g^2(x) \, \mathrm{d}x \right)^{\frac{1}{2}} \right)^2 
\geq \int_{a}^{b} f^2(x) \, \mathrm{d}x + \int_{a}^{b} g^2(x) \, \mathrm{d}x + 2 \int_{a}^{b} f(x) g(x) \, \mathrm{d}x.
\]

由此可得
\[
\left( \left( \int_{a}^{b} f^2(x) \, \mathrm{d}x \right)^{\frac{1}{2}} + \left( \int_{a}^{b} g^2(x) \, \mathrm{d}x \right)^{\frac{1}{2}} \right)^2 
\geq \int_{a}^{b} \left( f(x) + g(x) \right)^2 \, \mathrm{d}x.
\]

两边取平方根,即得
\[
\left( \int_{a}^{b} \left( f(x) + g(x) \right)^2 \, \mathrm{d}x \right)^{\frac{1}{2}} 
\leq \left( \int_{a}^{b} f^2(x) \, \mathrm{d}x \right)^{\frac{1}{2}} + \left( \int_{a}^{b} g^2(x) \, \mathrm{d}x \right)^{\frac{1}{2}}.
\]
\end{proof}

\subsubsection*{1.7.9}

求定积分
\[
I_n = \int_{0}^{\frac{\pi}{2}} \cos^n x \, \mathrm{d}x = \int_{0}^{\frac{\pi}{2}} \sin^n x \, \mathrm{d}x, \quad n = 0, 1, 2, \dots
\]

\begin{proof}
首先证明等式
\[
\int_{0}^{\frac{\pi}{2}} \sin^n x \, \mathrm{d}x = \int_{0}^{\frac{\pi}{2}} \cos^n x \, \mathrm{d}x.
\]

设 $x = \frac{\pi}{2} - t$,则 $\mathrm{d}x = -\mathrm{d}t$,当 $x = 0$ 时,$t = \frac{\pi}{2}$;当 $x = \frac{\pi}{2}$ 时,$t = 0$。于是
\[
\int_{0}^{\frac{\pi}{2}} \sin^n x \, \mathrm{d}x = \int_{\frac{\pi}{2}}^{0} \sin^n \left( \frac{\pi}{2} - t \right) (-\mathrm{d}t)
= \int_{0}^{\frac{\pi}{2}} \sin^n \left( \frac{\pi}{2} - t \right) \, \mathrm{d}t.
\]
由 $\sin\left(\frac{\pi}{2} - t\right) = \cos t$,得
\[
\int_{0}^{\frac{\pi}{2}} \sin^n x \, \mathrm{d}x = \int_{0}^{\frac{\pi}{2}} \cos^n t \, \mathrm{d}t = \int_{0}^{\frac{\pi}{2}} \cos^n x \, \mathrm{d}x.
\]

再由定积分分部积分公式,得
\[
I_n = \int_{0}^{\frac{\pi}{2}} \sin^n x \, \mathrm{d}x = \int_{0}^{\frac{\pi}{2}} \sin^{n-1} x \cdot \cos x \, \mathrm{d}x,
\]
令 $u = \sin^{n-1} x, v' = \cos x$,则 $u' = (n-1) \sin^{n-2} x \cos x$,$v = \sin x$,于是
\[
I_n = \sin^{n-1} x \sin x \Big|_{0}^{\frac{\pi}{2}} - \int_{0}^{\frac{\pi}{2}} \sin x \cdot (n-1) \sin^{n-2} x \cos x \, \mathrm{d}x.
\]
化简得
\[
I_n = (n-1) \int_{0}^{\frac{\pi}{2}} \sin^{n-2} x (1 - \sin^2 x) \, \mathrm{d}x,
\]
\[
I_n = (n-1) \int_{0}^{\frac{\pi}{2}} \sin^{n-2} x \, \mathrm{d}x - (n-1) \int_{0}^{\frac{\pi}{2}} \sin^n x \, \mathrm{d}x.
\]

令 $\int_{0}^{\frac{\pi}{2}} \sin^{n-2} x \, \mathrm{d}x = I_{n-2}$,则
\[
I_n = (n-1) I_{n-2} - (n-1) I_n,
\]
\[
I_n + (n-1) I_n = (n-1) I_{n-2},
\]
\[
I_n = \frac{n-1}{n} I_{n-2}, \quad (n \geq 2).
\]

这个等式叫作积分 $I_n$ 关于下标 $n$ 的递推公式。如果将 $n$ 换成 $n-2$,则有
\[
I_{n-2} = \frac{n-3}{n-2} I_{n-4}.
\]

同样地依次进行下去,直到 $I_n$ 的下标递减到 $0$ 或 $1$ 为止,于是
\[
I_{2m} = \frac{2m-1}{2m} \cdot \frac{2m-3}{2m-2} \cdot \dots \cdot \frac{1}{2} \cdot I_0,
\]
\[
I_{2m+1} = \frac{2m}{2m+1} \cdot \frac{2m-2}{2m-1} \cdot \dots \cdot \frac{2}{3} \cdot \frac{2}{1} \cdot I_1.
\]

其中当 $n = 0$ 时,$\sin^0 x = 1$,
\[
I_0 = \int_{0}^{\frac{\pi}{2}} \, \mathrm{d}x = \frac{\pi}{2}.
\]

因此,
\[
I_1 = \int_{0}^{\frac{\pi}{2}} \sin x \, \mathrm{d}x = -\cos x \Big|_{0}^{\frac{\pi}{2}} = 1.
\]

\[
I_n = \int_{0}^{\frac{\pi}{2}} \sin^n x \, \mathrm{d}x = \frac{n-1}{n} \cdot \frac{n-3}{n-2} \cdot \frac{n-5}{n-4} \cdots \frac{5}{6} \cdot \frac{3}{4} \cdot \frac{1}{2} \cdot \frac{\pi}{2}, \quad n \text{ 是偶数};
\]

\[
I_n = \int_{0}^{\frac{\pi}{2}} \sin^n x \, \mathrm{d}x = \frac{n-1}{n} \cdot \frac{n-3}{n-2} \cdot \frac{n-5}{n-4} \cdots \frac{6}{7} \cdot \frac{4}{5} \cdot \frac{2}{3} \cdot 1, \quad n \text{ 是奇数}.
\]

\end{proof}

\subsubsection*{1.7.10}
设 $(0, +\infty)$ 上的连续函数 $f(x)$ 满足
\[
f(x) = \ln x - \int_1^e f(x) \, dx,
\]
求
\[
I = \int_1^e f(x) \, dx.
\]

\begin{proof}
记
\[
\int_{1}^{e} f(x) \, \mathrm{d}x = a,
\]
则 $f(x) = \ln x - a$,于是
\[
a = \int_{1}^{e} f(x) \, \mathrm{d}x = \int_{1}^{e} \ln x \, \mathrm{d}x - a (e - 1),
\]
所以
\[
a = \frac{1}{e} \int_{1}^{e} \ln x \, \mathrm{d}x = \frac{1}{e} \left( x \ln x - x \right) \Big|_{1}^{e} = \frac{1}{e}.
\]
\end{proof}



\subsubsection*{1.7.11}
设函数 $f(x)$ 连续且满足
\[
\int_0^1 t f(2x - t) \, dt = \frac{1}{2} \arctan x^2, \quad f(1) = 1.
\]
求
\[
I = \int_1^2 f(x) \, dx.
\]

\begin{proof}
	由
\[
\int_{0}^{x} f(2x - t) \, \mathrm{d}t = \int_{x}^{2x} (2x - u) f(u) (-\mathrm{d}u),
\]
即
\[
\int_{0}^{x} f(2x - t) \, \mathrm{d}t = \int_{x}^{2x} (2x - u) f(u) \, \mathrm{d}u = 2x \int_{x}^{2x} f(u) \, \mathrm{d}u - \int_{x}^{2x} u f(u) \, \mathrm{d}u.
\]

得
\[
2x \int_{x}^{2x} f(u) \, \mathrm{d}u - \int_{x}^{2x} u f(u) \, \mathrm{d}u = \frac{1}{2} \arctan^2 x,
\]
等式两边对 $x$ 求导得
\[
2 \left[ 2x f(2x) + 2x \int_{x}^{2x} f(u) \, \mathrm{d}u - f(x) - 4x f(x) \right] + x f(x) = \frac{x}{1 + x^2},
\]
整理得
\[
2 \int_{x}^{2x} f(u) \, \mathrm{d}u - x f(x) = \frac{x}{1 + x^2}.
\]

取 $x = 1$ 得
\[
2 \int_{1}^{2} f(u) \, \mathrm{d}u - f(1) = \frac{1}{2},
\]
故
\[
\int_{1}^{2} f(x) \, \mathrm{d}x = \frac{3}{4}.
\]

\end{proof}

\subsubsection*{1.7.12}
设函数
\[
S(x) = \int_0^x |\cos t| \, dt,
\]
求
\[
\lim_{x \to +\infty} \frac{S(x)}{x}.
\]

\begin{proof}
    设 $n\pi < x \leq (n+1)\pi, n$ 为正整数,则 $\frac{x}{n} \to \pi$ 当 $x \to +\infty$。由
    \[
    \int_0^{n\pi} |\cos x| \, dx = n, \quad \int_0^x |\cos x| \, dx = 2n \leq \int_{n\pi}^x |\cos x| \, dx \leq \int_{n\pi}^\infty |\cos x| \, dx = \pi,
    \]
    可知
    \[
    2n \leq S(x) \leq 2n + \pi.
    \]
    因此,得
    \[
    \frac{2n}{x} \leq \frac{S(x)}{x} \leq \frac{2n + \pi}{x}.
    \]
    当 $x \to +\infty$ 时,$\frac{2n}{x} \to 0$ 和 $\frac{2n + \pi}{x} \to 0$,从而
    \[
    \lim_{x \to +\infty} \frac{S(x)}{x} = \frac{2}{\pi}.
    \]
\end{proof}

\subsubsection*{1.7.13}

设函数 $f(x)$ 在 $(0,+\infty)$ 上连续,证明
\[
\int_1^4 f\left( \frac{x}{2} + \frac{2}{x} \right) \frac{\ln x}{x} \, dx = \ln 2 \int_1^4 f\left( \frac{x}{2} + \frac{2}{x} \right) \frac{1}{x} \, dx.
\]

\begin{proof}
    令 $t = \frac{4}{x}$,则 $x = \frac{4}{t}$,$dx = -\frac{4}{t^2} dt$,于是
    \[
    \int_1^4 f\left( \frac{x}{2} + \frac{2}{x} \right) \ln \frac{x}{x} \, dx = \int_4^1 f\left( \frac{t}{2} + \frac{2}{t} \right) t \left( \ln 4 - \ln t \right) \left( -\frac{4}{t^2} \right) \, dt.
    \]
    \[
    = \int_1^4 f\left( \frac{x}{2} + \frac{2}{x} \right) \ln 4 - \ln x \, dx
    \]
    所以
    \[
    \int_1^4 f\left( \frac{x}{2} + \frac{2}{x} \right) \ln \frac{x}{x} dx = (\ln 2) \int_1^4 f\left( \frac{x}{2} + \frac{2}{x} \right) \, dx.
    \]
\end{proof}

\subsubsection*{1.7.14}

设函数 $f(x)$ 在 $[0,1]$ 上有二阶连续导数,满足
\[
f(0) = f(1) = 0, \quad f'(0) = 0, \quad f'(1) = 1.
\]
试证
\[
\int_0^1 \left( f''(x) \right)^2 dx \geq 4,
\]
并指出不等式中等号成立的条件。

\begin{proof}
    构造一个三次多项式 $p(x)$,满足 $p(0) = p(1) = 0, p'(0) = 0, p'(1) = 1$。于是有
    \[
    p(x) = kx^2(x-1), \quad 1 = p'(1) = k \cdot 1 \cdot (1-1) = k, \quad \text{得} \quad p(x) = x^3 - x^2, \quad p'(x) = 3x^2 - 2x, 
    \]
    \[
    \quad p''(x) = 6x - 2, \quad p^{(3)}(x) = 0.
    \]
    因此
    \[
    \int_0^1 [p''(x)]^2 dx = \int_0^1 (36x^2 - 24x + 4) dx = 12 - 12 + 4 = 4.
    \]

    当 $f(x) = x^3 - x^2$ 时,等号成立。考虑积分
    \[
    \int_0^1 \left( f''(x)^2 - [p''(x)]^2 \right) dx,
    \]
    有
    \[
    = \int_0^1 [f''(x) - p''(x)]^2 dx + 2 \int_0^1 f'(x) p^{(3)}(x) dx - 2 \int_0^1 p''(x) f^{(3)}(x) dx = 8
    \]

    所以
    \[
    \int_0^1 \left( f''(x) \right)^2 dx \geq \int_0^1 \left[ p''(x) \right]^2 dx = 4,
    \]
    且等号成立当 $f(x) = p(x)$ 时。再由 $f$ 与 $p$ 满足条件,得
    \[
    f(x) = p(x) = x^3 - x^2.
    \]
\end{proof}

\subsubsection*{1.7.15}

设函数 $f(x)$ 在 $[0,1]$ 上非负连续,且严格递增。对每个正整数 $n$,记
\[
F_n(x) = f(x)^n, \quad \text{设} \quad \theta_n \in [0,1] \quad \text{满足} \quad F_n(\theta_n) = \int_0^1 F_n(x) \, dx.
\]
证明
\[
\lim_{n \to \infty} \theta_n = 1.
\]

\begin{proof}
    由条件, 对每个 $n, F_n(x)$ 在 $[0,1]$ 上也都是非负、严格递增的连续函数。

    对 $\epsilon > 0$ ($\epsilon < \frac{1}{2}$),因为
    \[
    0 \leq \frac{f(1 - 2\epsilon)}{f(1 - \epsilon)} < 1 \quad \Rightarrow \quad \lim_{n \to \infty} \left( \frac{f(1 - 2\epsilon)}{f(1 - \epsilon)} \right)^n = 0,
    \]
    所以存在 $N > 0$,当 $n > N$ 时,
    \[
    \left( \frac{f(1 - 2\epsilon)}{f(1 - \epsilon)} \right)^n = \frac{F_n(1 - 2\epsilon)}{F_n(1 - \epsilon)} < \epsilon.
    \]
    从而又有
    \[
    F_n(1 - 2\epsilon) < \epsilon F_n(1 - \epsilon) < \int_{1 - \epsilon}^1 F_n(x) \, dx < \int_0^1 F_n(x) \, dx = F_n(\theta_n).
    \]
    再由 $F_n(x)$ 为严格递增,得知 $n > N$ 时满足
    \[
    1 - 2\epsilon < \theta_n < 1.
    \]
    这就证明得
    \[
    \lim_{n \to \infty} \theta_n = 1.
    \]
\end{proof}

% 第二部分:等距分划下的Riemann积分
\newpage
\section{第二部分:等距分划下的Riemann积分}

% 2.1 黎曼积分的定义
\subsection{黎曼积分的定义}
设 $f$ 是定义在 $[a,b]$ 上的一个函数。如果存在一个实数 $I$,使得任意 $\epsilon > 0$,存在 $\delta > 0$,对于 $[a,b]$ 的任意一个分割
\[
P : a = x_0 < x_1 < \cdots < x_{n-1} < x_n = b,
\]
只要其宽度
\[
\|P\| = \max_{1 \leq k \leq n} \Delta x_k < \delta, \quad \text{此处} \quad \Delta x_k = x_k - x_{k-1}, \ k = 1,2,\cdots,n,
\]
那么,在每个子区间 $[x_{k-1},x_k]$ 上任取一点 $\xi_k$,都有
\[
\left| \sum_{k=1}^n f(\xi_k)\Delta x_k - I \right| < \epsilon,
\]
我们就说 $f$ 在 $[a,b]$ 上可积,称为 $f$ 在 $[a,b]$ 上的黎曼积分,简称积分,记为
\[
\lim_{\|P\| \to 0} \sum_{k=1}^n f(\xi_k)\Delta x_k = \int_a^b f(x)\, \mathrm{d}x = I.
\]

% 2.2 问题
\subsection{问题}
如果将上述定义中的分割 $P$ 改成对区间 $[a,b]$ 实行 $n$ 等分,即要求所有 $\Delta x_k$ 都相等,这样得出的可积性以及积分的定义与原来的定义是否等价?为什么?

% 2.3 补充定义
\subsection{可积的第一第二充要条件}

% 2.3.1 等距分划
\subsubsection*{等距分划}
称点集 $P = \{x_0, x_1, \cdots, x_{n-1}, x_n\}$ 为 $[a, b]$ 的一个分割,如果满足条件:
\[
a = x_0 < x_1 < \cdots < x_{n-1} < x_n = b.
\]
记 $\Delta x_i = x_i - x_{i-1}, \ i = 1, \cdots, n$,并称 $\|P\| = \max_{1 \leq i \leq n} \{\Delta x_i\}$ 为分割 $P$ 的细度。

如果 $\Delta x_i = \frac{b-a}{n}, \ i = 1, \cdots, n$,则称 $P$ 为\textbf{等距分划}。

% 2.3.2 介点集与Riemann积分和
\subsubsection*{介点集与Riemann积分和}
设 $P = \{x_0, x_1, \cdots, x_{n-1}, x_n\}$ 为区间 $[a, b]$ 的一个分割。对每个子区间 $[x_{i-1}, x_i]$,任选 $\xi_i \in [x_{i-1}, x_i]$,则称 $\xi = \{\xi_i \mid i = 1, 2, \cdots, n\}$ 为从属于 $P$ 的一个 \textbf{介点集};并称和式
\[
\sum_{i=1}^n f(\xi_i) \Delta x_i \quad \text{或} \quad \sum_P f(\xi_i) \Delta x_i
\]
为 $f$ 在区间 $[a, b]$ 上的一个 \textbf{Riemann(积分)和}。

% 2.3.3 可积与定积分
\subsubsection*{可积与定积分}
设 $I$ 为实数,且有
\[
\lim_{\|P\| \to 0} \sum_{i=1}^n f(\xi_i) \Delta x_i = I,
\]
即 $\forall \epsilon > 0, \exists \delta > 0$,对 $\|P\| < \delta$ 的每个分割 $P$,以及对从属于 $P$ 的每个介点集 $\xi$,成立
\[
\left| \sum_{i=1}^n f(\xi_i) \Delta x_i - I \right| < \epsilon,
\]
则称函数 $f$ 在区间 $[a, b]$ 上 \textbf{Riemann 可积} 或简称 \textbf{可积},记为
\[
f \in R[a, b],
\]
并称 $I$ 为 $f$ 在区间 $[a, b]$ 上的 \textbf{Riemann 积分} 或 \textbf{定积分},简称积分,记为
\[
\int_a^b f(x) \, \mathrm{d}x = I, \quad \text{或其简化记号} \quad \int_a^b f = I.
\]
% 2.3.4 振幅与振幅面积
\subsubsection*{振幅与振幅面积}
为叙述可积的充分必要条件,需要引入以下概念。设函数 $f$ 在区间 $[a, b]$ 上有界,$P = \{x_0, x_1, \cdots, x_{n-1}, x_n\}$ 为 $[a, b]$ 的一个分割,对 $i = 1, \cdots, n$,记
\[
M_i = \sup\{f(x) \mid x \in [x_{i-1}, x_i]\} \quad \text{与} \quad m_i = \inf\{f(x) \mid x \in [x_{i-1}, x_i]\},
\]
称 $\omega_i = M_i - m_i$ 为 $f$ 在 $[x_{i-1}, x_i]$ 上的\textbf{振幅},$\sum_{i=1}^n \omega_i \Delta x_i$ 为 $f$ 的\textbf{振幅面积}。
% 2.3.5 可积的第一充分必要条件
\subsubsection*{可积的第一充分必要条件}
\textbf{命题2.3.5(可积的第一充分必要条件)}  

有界函数 $f \in R[a, b]$ 的充分必要条件是
\[
\lim_{\|P\| \to 0} \sum_{i=1}^n \omega_i \Delta x_i = 0.
\]
% 2.3.6 可积的第二充分必要条件
\subsubsection*{可积的第二充分必要条件}

\textbf{命题 2.3.6(可积的第二充分必要条件)}  

有界函数 $f \in R[a, b]$ 的充分必要条件是对每个 $\varepsilon > 0$,存在区间 $[a, b]$ 的一个分割 $P$,使成立
\[
\sum_P \omega_i \Delta x_i < \varepsilon.
\]

\textbf{这两个命题在许多教科书上都有具体的证明过程,在此不作证明直接使用。} 

证明过程可详见参考文献\cite{key3}\cite{key4}

% 2.4 证明
\subsection{Riemann积分引理}

\subsubsection*{引理叙述}
设 $f \in R[a, b]$ 且
\[
\int_a^b f = I,
\]
其充分必要条件是存在 $[a, b]$ 的一个分割序列 $\{P_k\}_{k \in \mathbb{N}_+}$,满足条件
\[
\lim_{k \to \infty} \|P_k\| = 0,
\]
使得
\[
\lim_{k \to \infty} \sum_{i=1}^{n_k} f(\xi_{k, i}) \Delta x_{k, i} = I,
\]
且极限值不依赖于介点集的选取。

\subsubsection*{引理证明}
必要性显然,下面证明充分性。

既然极限值不依赖于介点集的选取,那么我们有
\[
\lim_{k \to \infty} \sum_{i=1}^{n_k} \left( \sup_{(\xi_{k, i} \in [x_{k, i-1}, x_{k, i}])} f(\xi_{k, i}) \right) \Delta x_{k, i}
= \lim_{k \to \infty} \sum_{i=1}^{n_k} \left( \inf_{(\xi_{k, i} \in [x_{k, i-1}, x_{k, i}])} f(\xi_{k, i}) \right) \Delta x_{k, i} = I.
\]
两者相减得到
\[
\lim_{k \to \infty} \sum_{i=1}^{n_k} \omega_i \Delta x_{k, i} = 0,
\]
其中 $\omega_i = M_i - m_i$ 表示 $f$ 在 $[x_{k, i-1}, x_{k, i}]$ 上的振幅。

故根据可积的第一充分必要条件,$f \in R[a, b]$。由于
\[
\lim_{k \to \infty} \sum_{i=1}^{n_k} f(\xi_{k, i}) \Delta x_{k, i} = I,
\]
根据积分函数点集选取的任意性性质,有
\[
\int_a^b f(x) \, \mathrm{d}x = I.
\]

% 2.5 解答问题2.2
\subsection{问题2.2解答}

\subsubsection*{问题描述}
如果将黎曼积分定义中的分割 $P$ 改成对区间 $[a, b]$ 进行 $n$ 等分,即所有 $\Delta x_k$ 相等,这样得出的可积性以及积分的定义是否与原定义等价?为什么?

\subsubsection*{问题解答}
引理2.4.1表明Riemann积分的定义中分划的任意性要求可以降低,例如等距分划也是可以的。\cite{key1}
对于任意一个函数 $f$,若其在 $[a, b]$ 上 Riemann 可积,则对于任意分割序列 $\{P_k\}$,只要 $\|P_k\| \to 0$,都可以得出相同的积分值。因此,将分割改为等距分划不会改变函数的可积性以及积分值。

具体而言,设 $P_k$ 为等距分割,每个子区间长度为
\[
\Delta x = \frac{b-a}{n_k},
\]
根据 Riemann 积分定义,当 $n_k \to \infty$ 时,仍满足
\[
\lim_{n_k \to \infty} \sum_{i=1}^{n_k} f(\xi_{k, i}) \Delta x = \int_a^b f(x) \, \mathrm{d}x.
\]

因此,等距分划定义的黎曼积分与原定义等价。


\clearpage
\section{第三部分:离散动力系统的混沌现象}

\subsection{周期点、周期轨道与不动点}

假设 $I$ 是一个区间,函数 $f: I \to I$。对任意 $x \in I$,我们规定
\[
f^0(x) = x, \quad f^1(x) = f(x), \quad f^{n+1}(x) = f(f^n(x)), \quad n = 1, 2, 3, \cdots.
\]
这样得到的函数 $f^n: I \to I$ 称为 $f$ 的第 $n$ 次迭代($n = 0, 1, 2, \cdots$)。显然,$f^0$ 就是恒等映射,$f^1$ 就是 $f$ 自身。

对给定的 $x \in I$,我们考虑点列
\[
x, \, f(x), \, f^2(x), \, \cdots.
\]
如果有正整数 $m$ 使得 $f^m(x) = x$,则 $x$ 称为 $f$ 的一个\textbf{周期点},把 $m$ 称为 $x$ 的\textbf{一个周期}。

如果 $x$ 的最小周期是 $n$,则称 $x$ 是 $f$ 的一个$\mathbf{n-}$ \textbf{周期点}。这时点列
\[
x, \, f(x), \, f^2(x), \, \cdots, \, f^{m-1}(x)
\]
是由 $n$ 个互不相同的点组成的有限数列,称为 $x$ 的 $\mathbf{n-}$ \textbf{周期轨道}。

$f$ 的 $1$ 周期点也称为 $f$ 的\textbf{不动点}。

\subsection{问题}

\subsubsection{问题1(Brouwer 不动点定理)}

证明:如果 $I$ 是一个闭区间,$f: I \to I$ 连续,则 $f$ 必有不动点。

\subsubsection{问题2(构造2-周期点函数)}

试构造一个 $[0, 1]$ 上的连续函数,使得 $f$ 有 2-周期点。

\subsubsection{问题3(构造3-周期点函数)}

试构造一个 $[0, 1]$ 上的连续函数,使得 $f$ 有 3-周期点。

\subsubsection{问题4(Li-Yorke 定理)}

证明:如果 $f: I \to I$ 连续,且有 3-周期点,那么对任意正整数 $n$,$f$ 必有 $n$-周期点。

\subsection{两个补充定理}

\subsubsection{零点存在定理}

设 $f \in C[a, b]$,并满足条件 $f(a)f(b) < 0$,则存在点 $\xi \in (a, b)$ 使得 $f(\xi) = 0$。

\subsubsection{介值定理}

区间上的连续函数的值域必是区间(可缩为一点)。

\textbf{这两个定理在许多教科书上都有具体的证明方法,在此直接使用不作证明}。详细证明过程在参考文献\cite{key5}。

\subsection{问题解答}

\subsubsection*{3.4.1  问题1解答}

这是著名的Brouwer(布劳威尔)不动点定理的特例,下面给出一种证明。

原问题:如果 $I$ 是一个闭区间,$f: I \to I$ 连续,则 $f$ 必有不动点等价于以下命题

\textbf{Brouwer 不动点定理}
如果 $I = [a, b] \subseteq \mathbb{R}$,并且 $f: I \to I$ 是连续函数,那么 $f$ 至少有一个不动点。\cite{key8}


\textbf{解:} 
如果 $f(a) = a$ 或 $f(b) = b$,那么命题显然成立。否则,$f(a) > a$ 且 $f(b) < b$。

定义一个辅助函数:
\[
g(x) = f(x) - x, \quad x \in [a, b].
\]
显然,$g(a) = f(a) - a > 0$,而 $g(b) = f(b) - b < 0$。

又因为 $f$ 是连续函数,故 $g(x)$ 也是 $[a, b]$ 上的连续函数。因此,根据介值定理(Intermediate Value Theorem),存在 $x^* \in (a, b)$,使得
\[
g(x^*) = 0.
\]
由 $g(x^*) = f(x^*) - x^* = 0$ 可得
\[
f(x^*) = x^*.
\]
因此,$x^*$ 是 $f$ 的不动点。命题得证。

从而原问题 “如果 $I = [a, b] \subseteq \mathbb{R}$,并且 $f: I \to I$ 是连续函数,那么 $f$ 至少有一个不动点” 得证。\qed 

\subsubsection*{3.4.2  问题2解答}

原问题:试构造一个 $[0, 1]$ 上的连续函数,使得 $f$ 有 2-周期点。

\textbf{解:} 
构造函数 $f(x) = x^2 - 1$  

点 $x_0 = -1$ 是函数 $f(x) = x^2 - 1$ 的一个周期为 2 的2-周期点,因为
\[
f(-1) = 0 \quad \text{且} \quad f^2(-1) = -1.
\]
同样地,$x_0 = 0$ 也是函数 $f(x)$ 的一个周期为 2 的2-周期点。\qed 

\subsubsection*{3.4.3 问题3解答}

原问题:试构造一个 $[0, 1]$ 上的连续函数,使得 $f$ 有 3-周期点。

\textbf{解:} 
构造函数 $f(x) = 1 - \frac{1}{2}x - \frac{3}{2}x^2$

点 $x_0 = 1$ 是函数 $f(x) = 1 - \frac{1}{2}x - \frac{3}{2}x^2$ 的一个周期为 3 的3-周期点,因为
\[
f(1) = -1, \quad f^2(1) = 0, \quad \text{且} \quad f^3(1) = 1.
\]\qed 

\subsubsection*{3.4.4  问题4解答}

这是中国台湾数学家李天岩和美国数学家Yorke, J.A.于1975年发表在《美国数学月刊》的论文《周期3蕴涵混沌》(《Period three implies chaos》)提出。被普遍称为Li-Yorke Theorem,是Sharkovsky 定理的一种特殊变形,下面给出两种证明方法。

问题:如果 $f: I \to I$ 连续,且有 3-周期点,那么对任意正整数 $n$,$f$ 必有 $n$-周期点。

\textbf{方法一}
\begin{proof}
设 $x_1$ 是周期为 3 的点,且 $f(x_1) = x_2$ 和 $f(x_2) = x_3$。假设 $x_1 < x_2 < x_3$(其他可能的排列方式可采用类似的证明)。令
\[
I_1 = [x_1, x_2], \quad I_2 = [x_2, x_3].
\]
那么有
\[
f(I_1) = I_2, \quad f(I_2) = I_1 \cup I_2.
\]
由于 $I_2 \to I_2$,区间 $I_2$ 中存在一个不动点 $p_1$(周期为 1)。同时有 $I_1 \to I_2 \to I_1$,这意味着在 $I_1$ 中存在一个周期为 2 的周期点 $p_2$。这两个点显然与 $x_1, x_2, x_3$ 不同,因为它们的周期小于 3。

此外,对于任意给定的整数 $n > 3$,可以构造以下序列:
\[
I_1 \to I_2 \to I_2 \to \cdots \to I_2 \to I_1,
\]
其中区间 $I_2$ 出现了 $n-1$ 次。

这表明在 $I_1$ 中存在一个周期为 $n$ 的点 $p_n$,其满足以下条件:
\[
f^n(p_n) = p_n, \quad f^i(p_n) \in I_2 \quad \text{对于所有 } 0 < i \leq n-1.
\]
注意,$p_n \neq x_2$,因为 $f^2(x_2) \in I_1 \setminus I_2$。因此,
\[
f^i(p_n) \neq p_n \quad \text{对于所有 } 0 < i \leq n-1,
\]
这表明 $p_n$ 的周期正好为 $n$。

\end{proof}

\textbf{方法二}
\begin{proof}

\textbf{引理 1(其实为问题1)}设函数 $f: I \to I$ 连续,且 $J = [a, b] \subset I$。如果 $f(J) \subset J$,那么 $f$ 在 $J$ 上有一个不动点。

\textbf{解:} 由于 $f(J) \subset J$,可以找到 $c, d \in J$,使得 $f(c) = a$ 以及 $f(d) = b$。若 $c = a$ 或 $d = b$,那么 $a$ 或 $b$ 就是 $f$ 的不动点。如果不是这样,则说明 $c > a$ 且 $d < b$。这时,连续函数 $\varphi(x) = f(x) - x$ 满足
\[
\varphi(c) = f(c) - c = a - c < 0, \quad \varphi(d) = f(d) - d = b - d > 0.
\]
因此由零值定理可知,$\varphi$ 在 $c$ 与 $d$ 之间必有一个零点,这一点正是 $f$ 的不动点。
\qed

\textbf{引理 2} 设函数 $f: I \to I$ 连续,$J_1, J_2$ 是 $I$ 的两个闭子区间。如果 $f(J_1) \supset J_2$,那么必存在 $J_1$ 的闭子区间 $K$,使得 $f(K) = J_2$。

\textbf{解:} 设 $J_1 = [a, b], J_2 = [U, V]$,由于 $f(J_1) \supset J_2$,必存在 $u, v \in [a, b]$,使得 $f(u) = U, f(v) = V$。不妨设 $u < v$($u > v$ 的情形可类似处理)。令
\[
E = \{s : f(s) = U, \, u \leq s \leq v \}.
\]
因为 $f(u) = U$,故 $E$ 为非空集且有上界,因此 $E$ 必有上确界。记 $u^* = \sup E$,我们证明 $f(u^*) = U$。由上确界的定义知,对于任意的正整数 $n$,必有 $s_n \in E$,使得
\[
u^* - \frac{1}{n} < s_n \leq u^*.
\]
由此可得 $\lim s_n = u^*$。由于 $s_n \in E$,故 $f(s_n) = U$。令 $n \to \infty$,并利用 $f$ 的连续性,即得 $f(u^*) = U$。这就证明了 $u^* \in E$,而且 $u^* \neq v$(因为 $f(v) = V$)。有了 $u^*$ 之后,可以定义
\[
F = \{t : f(t) = V, \, u^* < t \leq v\}.
\]
$F$ 当然有下确界,记 $v^* = \inf F$。同理可证 $f(v^*) = V$。由此可知 $u^* \neq v^*$。记
\[
K = [u^*, v^*].
\]
那么对于任意的 $x \in (u^*, v^*)$,由于 $x > u^*$,必有 $f(x) \neq U$;又因 $x < v^*$,必有 $f(x) \neq V$,故由介值定理,对任意的 $\eta \in (U, V)$,必有 $\xi \in [u^*, v^*]$,使得 $f(\xi) = \eta$。这就证明了

\begin{equation}
f([u^*, v^*]) \supset [U, V]. \tag{1}
\end{equation}

为了证明 $f([u^*, v^*]) \subset [U, V]$,必须证明:对任意的 $x < U$(或 $x > V$),不可能存在 $s \in [u^*, v^*]$,使得 $f(s) = x$。如果有这样的 $x$,我们再取一点 $x' \in (U, V)$。根据式(1),存在 $s' \in [u^*, v^*]$,使得 $f(s') = x'$,那么由于 $f(s') = x'$,而且 $x < U < x'$,故由介值定理,必有 $\xi$ 介于 $s'$ 与 $s$ 之间,使得 $f(\xi) = U$。由于 $\xi$ >$ U$,这是不可能的。这就证明了
\begin{equation}
f([u^*, v^*]) \subset [U, V]. \tag{2}
\end{equation}

综合式(1)与(2),即得 $f(K) = [U, V]$。
\qed

\textbf{引理 3} 设函数 $f: I \to I$ 连续,$J_0, J_1, \cdots, J_{n-1}$ 是 $I$ 的 $n$ 个闭子区间。

如果
\[
f(J_0) \supset J_1, \, f(J_1) \supset J_2, \, \cdots, \, f(J_{n-2}) \supset J_{n-1}, \, f(J_{n-1}) \supset J_0,
\]
那么:

\quad(1)存在 $x_0 \in J_0$,使得 $f^n(x_0) = x_0$;

\quad(2)$f(x_0) \in J_1, \, f^2(x_0) \in J_2, \, \cdots, \, f^{n-1}(x_0) \in J_{n-1}$。

用一句通俗的话来说,当 $j$ 从 $0$ 跑过 $1, 2, \cdots, n-1$ 时,$f(x_0)$ 依次地“拜访” $J_0, J_1, \cdots, J_{n-1}$,最后仍然回到 $x_0$。

\textbf{解:} 因为 $f(J_{n-1}) \supset J_0$,由引理2 知,有一个闭子区间 $K_{n-1} \subset J_{n-1}$,使得 $f(K_{n-1}) = J_0$。类似地,因 $f(J_{n-2}) \supset J_{n-1} \supset K_{n-1}$,又可以找到一个闭子区间 $K_{n-2} \subset J_{n-2}$,使得 $f(K_{n-2}) = K_{n-1}$。同理,可以找到一个闭子区间 $K_1 \subset J_1$,使得 $f(K_1) = K_2$。最后,存在在 $J_0$ 的闭子区间 $K_0$,使得 $f(K_0) = K_1$。

因此,我们看到
\[
\begin{array}{l}
f(K_0) = K_1, \\
f^2(K_0) = K_2, \\
f^3(K_0) = K_3, \\
\vdots \\
f^{n-1}(K_0) = K_{n-1}, \\
f^n(K_0) = f(K_{n-1}) = J_0 \supset K_0.
\end{array}
\]
对函数 $f^n$ 运用引理1,我们可以找到一点 $x_0 \in K_0 \subset J_0$,使得 $f^n(x_0) = x_0$。很显然,我们有 $f^k(x_0) \in K_k \subset J_k \ (k = 1, 2, \cdots, n-1)$。因此,引理3得证。
\qed 

\quad \textbf{现在可以来证明问题4。}

根据假定,设 $\eta$ 是 $f$ 的一个 3 周期点,那么 $\eta, f(\eta), f^2(\eta)$ 构成 $\eta$ 的 3 周期轨。不妨设
\[
\eta < f(\eta) < f^2(\eta).
\]
为简单起见,设 $a = \eta, \beta = f(\eta), \gamma = f^2(\eta)$,于是有
\[
f(a) = \beta, \, f(\beta) = \gamma, \, f(\gamma) = a.
\]
记 $H = [a, \beta], K = [\beta, \gamma]$。由于 $f(a) = \beta, f(\beta) = \gamma$,故由介值定理,知
\begin{equation}
f(H) \supset K. \tag{3}
\end{equation}

又因 $f(\beta) = \gamma, f(\gamma) = a$,仍由介值定理,知
\begin{equation}
f(K) \supset [a, \gamma] = H \cup K. \tag{4}
\end{equation}

现在来证明,对于任意的 $n \in \mathbb{N}^*$,$f^n$ 必有 n 周期点。当 $n = 1$ 时,由式 (4),知 $f(K) \supset K$,故由引理 1,$f$ 在 $K$ 上有一个不动点,即 1 周期点。再设 $n = 2$,由式 (4),知
\[
f(K) \supset H.
\]
由式 (3),知 $f(H) \supset K$。于是由引理 3 知,存在一点 $x_0 \in K$,使得
\[
f^2(x_0) = x_0, \, f(x_0) \in H.
\]
我们证明 2 是 $x_0$ 的最小周期。若 $f(x_0) = x_0$,那么 $x_0 \in H \cap K = \{\beta\}$,即 $x_0 = \beta$,这就导致
\[
f(x_0) = f(\beta) = \gamma > \beta = x_0.
\]
的矛盾。现设 $n > 3$,记
\[
J_0 = J_1 = \cdots = J_{n-2} = K, \, J_{n-1} = H.
\]
从式 (4),知 $f(J_j) \supset J_{j+1} \ (j = 0, 1, \cdots, n-2)$。又从式 (3),有 $f(J_{n-1}) \supset J_0$,即引理 3 的要求都满足。因此有一点 $x_0 \in J_0 = K$,使得$f^n(x_0) = x_0$,且
\begin{equation}
f^j(x_0) \in J_j \quad (j = 1, 2, \cdots, n-1).\tag{5}
\end{equation}

现在证明 $n$ 是 $x_0$ 的最小周期。否则,存在 $k < n$,使得 $f^k(x_0) = x_0$,于是 $x_0, f(x_0), \cdots, f^{k-1}(x_0)$ 构成 $x_0$ 的 k 周期轨。由于 $n > 1 \Rightarrow n > 2 \Rightarrow k - 1$,所以 $f^{n-1}(x_0)$ 必是
\[
x_0, f(x_0), \cdots, f^{n-2}(x_0)
\]
中的一个。由式 (5) 知,它们都在 $K$ 中。但是 $f^{n-1}(x_0) \in J_{n-1} = H$。这说明 $f^{n-1}(x_0) \in K \cap H = \{\beta\}$,因此 $x_0 = f^{n}(x_0) = f(\beta) = \gamma$,而
\[
a = f(\gamma) = f(x_0) = f(\beta) = \gamma \Rightarrow K = [\beta, \gamma].
\]
这是不可能的。

这样就完全证明了问题4。

\end{proof}

\quad Li-Yorke定理告诉我们,看似简单的映射可能表现出非常复杂的动态行为。反之,也存在一些情形,映射表现得极为简单。
% 文献部分

\newpage 
% 修改标题为“参考文献”
\addcontentsline{toc}{section}{参考文献}
\nocite{*}

% Print all references, whether cited or not
\printbibliography

\end{document} 
